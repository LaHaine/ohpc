\subsection{Related Work} \label{sec:related_work}

As highlighted in the previous section, the installation, deployment and
configuration of HPC clusters can be tedious work.  Consequently, a variety of
open-source and proprietary cluster management tools have been created that
provide different approaches to help address this problem.  In the open-source
arena, two widely known solutions have been Rocks and OSCAR.  Rocks
\cite{rocks2003,rocks_url} is an open-source Linux-based clustering solution
that aims to reduce the complexity of building HPC clusters.  It provides a
combined bundle of an underlying CentOS distribution with additional software
components covering the different administrative needs of a cluster,
and offers a GUI to help administrators walk through
different steps of the installation procedure.  All cluster services and tools
are installed and configured during the initial installation of the front-end
with no need to download and %manual
configure other external
packages. Furthermore, with the use of Rolls \cite{rolls2004}, the
administrator can customize the base installation
%of packages
with additional
optional software that integrates seamlessly
%and automatically
into the management and
packaging mechanisms of the base software.
%It makes use of Kickstart and RPM to manage the distribution of node file
%system and provides component based configuration method to realize module
%reuse.
Rolls
%appliances
exist to deploy Rocks in alternative environments
%large-scale environments different
%than HPC
such as sensor networks \cite{rolls_sensors2012} and clouds \cite{rolls_cloud2011}.
The most recent version of Rocks, version 6.2, was released in May 2015 and is
based on CentOS~(v6.6).

OSCAR \cite{oscar2001,oscar_url} (Open Source Cluster Application Resources) is
a fully integrated and easy to install software bundle designed for high
performance cluster computing.
%The fundamental function of OSCAR is to build
%and maintain clusters.
OSCAR follows a different methodology than Rocks; once
the front-end is installed and booted, the cluster building components are
downloaded and installed through tools that try to simplify the complexity of
the different administrative tasks.  There have been some variations of OSCAR
based on the same framework to cover different types of cluster environments
such as Thin-Oscar for diskless platforms, and HA-Oscar to support High
Availability. OSCAR has previously been supported on multiple Linux distributions such as
CentOS and Debian. However,
%even if OSCAR and its variations are still being
%used in production,
the project is no longer actively maintained and the latest
version 6.1.1 was released in 2011.
%OSCAR also provides a GUI for installation and configuration purposes.  OSCAR
%is based upon a virtual image of the target machine using System Installation
%Suite (SIS). There is also an OSCAR database (ODA) that stores cluster
%information, a parallel distributed “shell” tool set called C3 and an
%environment management facility called Env-Switcher.

A common issue that arises in the design of system management tool kits is the
inevitable trade-off between ease-of-use and customization flexibility. Rocks
has adopted a more turn-key approach which necessitates the need for some level
of embedded configuration management and Rocks leverages a hierarchical XML
schema. As will be discussed further in \S\ref{sec:repo_enable}, OpenHPC is
providing an HPC focused software repository and adopts a more building-block
approach. Consequently, it expects a certain level of expertise from the
administrator, but is intended to offer a greater choice of software
components, promote flexibility for use in a variety of system environments
and scales, be compatible with multiple Linux distributions, and be
interoperable with standalone configuration management systems.
%It is currently focused only on HPC environments.
Furthermore, as
a community effort, OpenHPC is supported and maintained by a group of vendors,
research centers and laboratories that share common goals of minimizing
duplicated effort and sharing of best practices.

Several end-user oriented projects also exist to mitigate the complexity of HPC
and scientific software management including EasyBuild~\cite{easybuild2012} and
Spack~\cite{spack2015}.  Both of these systems provide convenient methods for
building and installing many common HPC software packages. With similar goals,
OpenHPC differs in scope and process. We aim to provide a complete cluster
software stack capable of provisioning and administering a system in addition
to user-space development libraries. OpenHPC also seeks to leverage standard
Linux tools and practices to install and maintain software. Both EasyBuild and
Spack are currently packaged in the OpenHPC distribution to allow users to
further extend and customize their environment.
