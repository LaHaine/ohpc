Prior to booting the {\em compute} hosts, we configure them to use PXE as their
next boot mode. After the initial PXE, ensuing boots will return to using the default boot device
specified in the BIOS.

% begin_ohpc_run
% ohpc_comment_header Set nodes to netboot \ref{sec:boot_computes}
\begin{lstlisting}[language=bash,keywords={},upquote=true]
[sms](*\#*) nodesetboot compute network
\end{lstlisting}
% end_ohpc_run

At this point, the {\em master} server should be able to boot the newly defined
compute nodes. This is done by using the \texttt{nodepower} \Confluent{} command
leveraging IPMI protocol set up during the the {\em compute} node definition
in \S~\ref{sec:confluent_add_nodes}. The following power cycles each of the
desired hosts.


% begin_ohpc_run
% ohpc_comment_header Boot compute nodes
\begin{lstlisting}[language=bash,keywords={},upquote=true]
[sms](*\#*) nodepower compute boot
\end{lstlisting}
% end_ohpc_run

Once kicked off, the boot process should take about 5-10
minutes (depending on BIOS post times).  You can monitor the
provisioning by using the \texttt{nodeconsole} command, which displays serial console
for a selected node. Note that the escape sequence
is \texttt{CTRL-e c .} typed sequentially.

Successful provisioning can be verified by a parallel command on the compute
nodes. The \Confluent{}-provided
\texttt{nodeshell} command, which uses \Confluent{} node names and groups.  
For example, to run a command on
the newly imaged compute hosts using \texttt{nodeshell}, execute the following:

\begin{lstlisting}[language=bash]
[sms](*\#*) nodeshell compute uptime
c1:  12:56:50 up 14 min,  0 users,  load average: 0.00, 0.01, 0.04
c2:  12:56:50 up 13 min,  0 users,  load average: 0.00, 0.02, 0.05
c3:  12:56:50 up 14 min,  0 users,  load average: 0.00, 0.02, 0.05
c4:  12:56:50 up 14 min,  0 users,  load average: 0.00, 0.01, 0.04
\end{lstlisting}
