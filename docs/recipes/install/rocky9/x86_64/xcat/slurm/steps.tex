\documentclass[letterpaper]{article}
\usepackage{common/ohpc-doc}
\setcounter{secnumdepth}{5}
\setcounter{tocdepth}{5}

% Include git variables
\input{vc.tex}

% Define Base OS and other local macros
\newcommand{\baseOS}{Rocky 8.4}
\newcommand{\OSRepo}{Rocky\_8.4}
\newcommand{\OSTree}{CentOS\_8}
\newcommand{\OSTag}{el8}
\newcommand{\baseos}{rocky8.4}
\newcommand{\baseosshort}{rocky8}
\newcommand{\provisioner}{xCAT}
\newcommand{\provheader}{xCAT (stateless)}
\newcommand{\rms}{SLURM}
\newcommand{\rmsshort}{slurm}
\newcommand{\arch}{x86\_64}
\newcommand{\installimage}{netboot}

% Define package manager commands
\newcommand{\pkgmgr}{dnf}
\newcommand{\addrepo}{wget -P /etc/yum.repos.d}
\newcommand{\chrootaddrepo}{wget -P \$CHROOT/etc/yum.repos.d}
\newcommand{\clean}{dnf clean expire-cache}
\newcommand{\chrootclean}{dnf --installroot=\$CHROOT clean expire-cache}
\newcommand{\install}{dnf -y install}
\newcommand{\chrootinstall}{dnf -y --installroot=\$CHROOT install}
\newcommand{\groupinstall}{dnf -y groupinstall}
\newcommand{\groupchrootinstall}{dnf -y --installroot=\$CHROOT groupinstall}
\newcommand{\remove}{dnf -y remove}
\newcommand{\upgrade}{dnf -y upgrade}
\newcommand{\chrootupgrade}{dnf -y --installroot=\$CHROOT upgrade}
\newcommand{\tftppkg}{syslinux-tftpboot}
\newcommand{\beegfsrepo}{https://www.beegfs.io/release/beegfs\_7.2.1/dists/beegfs-rhel8.repo}

% boolean for os-specific formatting
\toggletrue{isCentOS}
\toggletrue{isCentOS_ww_slurm_x86}
\toggletrue{isSLURM}
\toggletrue{isx86}
\toggletrue{isxCAT}
\toggletrue{isCentOS_x86}

\begin{document}
\graphicspath{{common/figures/}}
\thispagestyle{empty}

% Title Page
% Title page and running header definition

\lhead{ \small {\color{logodarkgrey}\fontfamily{phv}\selectfont { Install Guide
    (v\OHPCVersion{})}:  {\baseOS{}/\arch{} + \provheader{} + \rms{}} } \vspace*{0pt} }

{\hspace*{4in} \includegraphics[width=1.7in]{ohpc_logo_blue.pdf}}

\vspace*{2cm}
\noindent {\LARGE \color{logodarkgrey} \fontfamily{phv}\selectfont OpenHPC (v\OHPCVersion{})} \vspace*{0.1cm} \\
\noindent {\LARGE \color{logodarkgrey} \fontfamily{phv}\selectfont Cluster Building Recipes} \\

{\color{logoblue}\noindent\rule{6.15in}{1.2pt}} \\

\noindent {\Large \color{logodarkgrey} \fontfamily{phv}\selectfont \baseOS{} Base OS} \\

\noindent{\Large\color{logodarkgrey}\fontfamily{phv}\selectfont{\provisioner{}/\rms{}
Edition for Linux*} (\arch{})} \\

{\color{logoblue}\noindent\rule{6.15in}{1.2pt}} \\ \vspace{0.2cm}

\iftoggleverb{isxCATstateful}
\vspace*{-0.2cm}
\noindent{\large\color{logodarkgrey}\fontfamily{phv}\selectfont{Stateful Provisioning}}
\fi

\vspace*{2.5in}

\noindent \includegraphics[width=0.75in]{3x_icon.png}

%\vspace*{.5in}
\noindent{\small \color{black} Document Last Update: \VCDateISO} \vspace*{0.1cm} \\
{\small \color{black} Document Revision: \VCRevision} \\ \vspace*{0.1cm}

% Disclaimer
\newpage

\vspace*{3.0cm}
\noindent {\Large \color{logoblue} \fontfamily{phv}\selectfont Legal Notice} \\

\vspace*{0.5cm}

\noindent Copyright {\small\copyright} 2016-2023, OpenHPC, a Linux Foundation
Collaborative Project. All rights reserved. \\

\vspace*{0.1cm}

\noindent \begin{tabular}{cp{10cm}}
\raisebox{-.75\height}{\includegraphics[width=0.22\textwidth]{cc_by}} &
This documentation is licensed under the Creative Commons Attribution 4.0 International
License. To view a copy of this license, visit
\href{http://creativecommons.org/licenses/by/4.0}{\color{blue}{http://creativecommons.org/licenses/by/4.0}}. \\
\end{tabular}


\vspace*{1.5cm}

{\footnotesize

\noindent Intel, the Intel logo, and other Intel marks are trademarks of Intel
Corporation in the U.S. and/or other countries. \\
\iftoggleverb{ispbs}
\noindent Altair, the Altair logo, OpenPBS, and other Altair marks are
trademarks of Altair Engineering, Inc. in the U.S. and/or other countries. \\
\fi
\noindent *Other names and brands may be claimed as the property of others. \\



}


\newpage
\tableofcontents
\newpage

% Introduction  --------------------------------------------------

\section{Introduction} \label{sec:introduction}
% begin_ohpc_run
% ohpc_validation_comment -----------------------------------------------------------------------------------------
% ohpc_validation_comment  Example Installation Script Template
% ohpc_validation_comment
% ohpc_validation_comment  This convenience script encapsulates command-line instructions highlighted in
% ohpc_validation_comment  an OpenHPC Install Guide that can be used as a starting point to perform a local
% ohpc_validation_comment  cluster install beginning with bare-metal. Necessary inputs that describe local
% ohpc_validation_comment  hardware characteristics, desired network settings, and other customizations
% ohpc_validation_comment  are controlled via a companion input file that is used to initialize variables
% ohpc_validation_comment  within this script.
% ohpc_validation_comment
% ohpc_validation_comment  Please see the OpenHPC Install Guide(s) for more information regarding the
% ohpc_validation_comment  procedure. Note that the section numbering included in this script refers to
% ohpc_validation_comment  corresponding sections from the companion install guide.
% ohpc_validation_comment -----------------------------------------------------------------------------------------
% ohpc_validation_newline

% ohpc_command inputFile=${OHPC_INPUT_LOCAL:-/opt/ohpc/pub/doc/recipes/BOSSHORT/input.local}
% ohpc_validation_newline
% ohpc_command if [ ! -e ${inputFile} ];then
% ohpc_command    echo "Error: Unable to access local input file -> ${inputFile}"
% ohpc_command    exit 1
% ohpc_command else
% ohpc_command    . ${inputFile} || { echo "Error sourcing ${inputFile}"; exit 1; }
% ohpc_command fi

% ohpc_validation_newline
% ohpc_validation_comment ---------------------------- Begin OpenHPC Recipe ---------------------------------------
% ohpc_validation_comment Commands below are extracted from an OpenHPC install guide recipe and are intended for
% ohpc_validation_comment execution on the master SMS host.
% ohpc_validation_comment -----------------------------------------------------------------------------------------

% end_ohpc_run

This guide presents a simple cluster installation procedure using components
from the \OHPC{} software stack. \OHPC{} represents an aggregation of a number
of common ingredients required to deploy and manage an HPC Linux* cluster
including provisioning tools, resource management, I/O clients, development
tools, and a variety of scientific libraries. These packages have been
pre-built with HPC integration in mind while conforming to common \Linux{}
distribution standards.
The documentation herein is intended to
be reasonably generic, but uses the underlying motivation of a small, 4-node
\iftoggleverb{isxCATstateful} stateful \else stateless \fi
cluster installation to define a step-by-step process. Several
optional customizations are included and the intent is that these collective
instructions can be modified as needed for local site customizations.
 \\

\noindent {\bf Base Linux Edition}: this edition of the guide highlights
installation without the use of a companion configuration management system and
directly uses distro-provided package management tools for component
selection. The steps that follow also highlight specific changes to system
configuration files that are required as part of the cluster install
process.
%Other editions of this guide provide similar install steps when using
%specific configuration management systems that can simplify the installation
%and configuration process.

\input{common/audience}
%\noindent {\bf Requirements/Assumptions}:
\subsection{Requirements/Assumptions}
This installation recipe assumes the availability of a single head node {\em
 master}, and four {\em compute} nodes. The {\em master} node serves as the
overall system management server (SMS) and is provisioned with \baseOS{} and is
subsequently configured to provision the remaining {\em compute} nodes with
\provisioner{} in a
\iftoggleverb{isxCATstateful} stateful \else stateless \fi
configuration. The terms {\em master} and SMS are
used interchangeably in this guide. For power management, we assume that
the compute node baseboard management controllers (BMCs) are available via IPMI
from the chosen master host. For file systems, we assume that the chosen master
server will host an \NFS{} file system that is made available to the compute
nodes.
\iftoggleverb{isx86}
Installation information is also discussed to optionally mount a
parallel file system and in this case, the parallel file system is assumed to
exist previously.
\fi

\begin{figure}[hbt]
\center
\iftoggleverb{isx86}
\includegraphics[width=0.85\linewidth]{ohpc-arch-small3.pdf}
\fi
\iftoggleverb{isaarch}
\includegraphics[width=0.85\linewidth]{ohpc-arch-small-eth.pdf}
\fi
\vspace*{-0.2cm}
\caption{Overview of physical cluster architecture.} \label{fig:physical_arch}
\end{figure}
\mbox{}

\vspace*{0.5cm}

An outline of the physical architecture discussed is shown in
Figure~\ref{fig:physical_arch} and highlights the high-level networking
configuration. The {\em master} host requires at least two Ethernet interfaces
with {\em eth0} connected to the local data center network and {\em eth1} used
to provision and manage the cluster backend (note that these interface names
are examples and may be different depending on local settings and OS
conventions). Two logical IP interfaces are expected to each compute node: the
first is the standard Ethernet interface that will be used for provisioning and
resource management. The second is used to connect to each host's BMC and is
used for power management and remote console access. Physical connectivity for
these two logical IP networks is often accommodated via separate cabling and
switching infrastructure; however, an alternate configuration can also be
accommodated via the use of a shared NIC, which runs a packet filter to divert
management packets between the host and BMC.

\iftoggleverb{isx86}
In addition to the IP networking, there is an optional high-speed network
(\InfiniBand{} or \OmniPath{} in this recipe) that is also connected to each of the
hosts. This high speed network is used for application message passing and
optionally for parallel file system connectivity as well (e.g. to
existing \Lustre{} or BeeGFS storage targets).
\fi

% -*- mode: latex; fill-column: 120; -*-

\subsection{Inputs} \label{sec:inputs}
As this recipe details installing a cluster starting from bare-metal, there is a requirement to define IP addresses and
gather hardware MAC addresses in order to support a controlled provisioning process. These values are necessarily unique
to the hardware being used, and this document uses variable substitution (\texttt{\$\{variable\}}) in the command-line
examples that follow to highlight where local site inputs are required. A summary of the required and optional variables
used throughout this recipe are presented below. Note that while the example definitions above correspond to a small
4-node compute subsystem, the compute parameters are defined in array format to accommodate logical extension to larger
node counts. \\

\vspace*{0.2cm}
\begin{tabular}{@{}>{\textbullet}l p{7cm} l}
& \texttt{\$\{sms\_name\}} & {\small \# Hostname for SMS server} \\
& \texttt{\$\{sms\_ip\}} & {\small \# Internal IP address on SMS server}  \\
\iftoggleverb{isxCAT}
& \texttt{\$\{domain\_name\}} & {\small \# Local network domain name}  \\
\fi
& \texttt{\$\{sms\_eth\_internal\}} & {\small \# Internal Ethernet interface on SMS} \\
\iftoggleverb{isWarewulf}
& \texttt{\$\{eth\_provision\}} & {\small \# Provisioning interface for computes} \\
\fi
& \texttt{\$\{internal\_netmask\}} & {\small \# Subnet netmask for internal network} \\
& \texttt{\$\{ntp\_server\}} & {\small \# Local ntp server for time synchronization} \\
& \texttt{\$\{bmc\_username\}} & {\small \# BMC username for use by IPMI} \\
& \texttt{\$\{bmc\_password\}} & {\small \# BMC password for use by IPMI} \\
& \texttt{\$\{num\_computes\}} & {\small \# Total \# of desired compute nodes} \\
& \texttt{\$\{c\_ip[0]\}}, \, \texttt{\$\{c\_ip[1]\}}, ... & {\small \# Desired compute node addresses} \\
& \texttt{\$\{c\_bmc[0]\}}, \texttt{\$\{c\_bmc[1]\}}, ... & {\small \# BMC addresses for computes} \\
& \texttt{\$\{c\_mac[0]\}}, \texttt{\$\{c\_mac[1]\}}, ... & {\small \# MAC addresses for computes} \\
& \texttt{\$\{c\_name[0]\}}, \texttt{\$\{c\_name[1]\}}, ... & {\small \# Host names for computes} \\
& \texttt{\$\{compute\_regex\}} & {\small \# Regex matching all compute node names (e.g. ``c*'')} \\
& \texttt{\$\{compute\_prefix\}} & {\small \# Prefix for compute node names (e.g. ``c'')} \\
\iftoggleverb{isxCAT}
& \texttt{\$\{iso\_path\}} & {\small \# Directory location of OS iso for \xCAT{} install} \\
\nottoggle{isxCATstateful}
{& \texttt{\$\{synclist\}} & {\small \# Filesystem location of \xCAT{} synclist file} \\}
\fi
\iftoggleverb{isxCATstateful}
& \texttt{\$\{ohpc\_repo\_dir\}} & {\small \# Directory location of local \OHPC{} repository mirror} \\
& \texttt{\$\{epel\_repo\_dir\}} & {\small \# Directory location of local EPEL repository
mirror} \\
\fi
\end{tabular}

\vspace*{0.2cm}
\noindent {Optional:}
\vspace*{0.1cm}

\begin{tabular}{@{}>{\textbullet}l p{7cm} l}
\iftoggleverb{isx86}
& \texttt{\$\{sysmgmtd\_host\}} & {\small \# BeeGFS System Management host name} \\
& \texttt{\$\{mgs\_fs\_name\}} & {\small \# Lustre MGS mount name} \\
& \texttt{\$\{sms\_ipoib\}} & {\small \# IPoIB address for SMS server} \\
& \texttt{\$\{ipoib\_netmask\}} & {\small \# Subnet netmask for internal IPoIB} \\
& \texttt{\$\{c\_ipoib[0]\}}, \texttt{\$\{c\_ipoib[1]\}}, ... & {\small \# IPoIB addresses for computes} \\
\fi
\iftoggleverb{isWarewulf}
& \texttt{\$\{kargs\}} & {\small \# Kernel boot arguments} \\
\fi
\end{tabular}




% begin_ohpc_run
% ohpc_validation_newline
% ohpc_validation_comment Verify OpenHPC repository has been enabled before proceeding
% ohpc_validation_newline
% ohpc_command dnf repolist | grep -q OpenHPC
% ohpc_command if [ $? -ne 0 ];then
% ohpc_command    echo "Error: OpenHPC repository must be enabled locally"
% ohpc_command    exit 1
% ohpc_command fi
% end_ohpc_run

% Base Operating System --------------------------------------------

\section{Install Base Operating System (BOS)}
In an external setting, installing the desired BOS on a
{\em master} SMS host typically involves booting from a DVD ISO image on a new
server. With this approach, insert the \baseOS{} DVD, power cycle the host, and
follow the distro provided directions to install the BOS on your chosen {\em
master} host.  Alternatively, if choosing to use a pre-installed server, please
verify that it is provisioned with the required \baseOS{} distribution. \\

\ifnottoggleverb{isWarewulf4}
Prior to beginning the installation process of \OHPC{} components, several additional
considerations are noted here for the SMS host configuration. First,
the installation recipe herein assumes that
the SMS host name is resolvable locally. Depending on the manner in which you
installed the BOS, there may be an adequate entry already defined
in \path{/etc/hosts}. If not, the following addition can be used to identify
your SMS host.
\begin{lstlisting}[language=bash,keywords={}]
[sms](*\#*) echo ${sms_ip} ${sms_name} >> /etc/hosts
\end{lstlisting}
\fi

While it is theoretically possible to enable SELinux on a cluster provisioned
with \provisioner{},
doing so is beyond the scope of this document. Even the use of
permissive mode can be problematic and we therefore recommend disabling SELinux on the {\em
master} SMS host. If SELinux components are installed locally,
the \texttt{selinuxenabled} command can be used to determine if SELinux is
currently enabled. If enabled, consult the distro documentation for information
on how to disable. \\

Finally, provisioning services rely on DHCP, TFTP, and HTTP network protocols.
Depending on the local BOS configuration on the SMS host, default firewall
rules may prohibit these services. Consequently, this recipe assumes that the
local firewall running on the SMS host is disabled (it is still recommended to
have additional security boundaries like a firewall to protect the cluster from
the Internet). If installed, the default firewall service can be disabled as
follows:


%\clearpage
% begin_ohpc_run
% ohpc_validation_newline
% ohpc_validation_comment Disable firewall
\begin{lstlisting}[language=bash,keywords={}]
[sms](*\#*) systemctl disable firewalld
[sms](*\#*) systemctl stop firewalld
\end{lstlisting}
% end_ohpc_run

% ------------------------------------------------------------------

\section{Install \OHPC{} Components} \label{sec:basic_install}
\input{common/install_ohpc_components_intro.tex}

\subsection{Enable \OHPC{} repository for local use} \label{sec:enable_repo}
To begin, enable use of the \OHPC{} repository by adding it to the local list
of available package repositories. Note that this requires network access from
your {\em master} server to the \OHPC{} repository, or alternatively, that
the \OHPC{} repository be mirrored locally.  In cases where network external
connectivity is available, \OHPC{} provides an \texttt{ohpc-release} package
that includes GPG keys for package signing and enabling the repository.  The
example which follows illustrates installation of the \texttt{ohpc-release}
package directly from the \OHPC{} build server.

\iftoggleverb{isCentOS}
% CentOS
\begin{lstlisting}[language=bash,keywords={},basicstyle=\fontencoding{T1}\fontsize{7.6}{10}\ttfamily,
	literate={VER}{\OHPCVerTree{}}1 {OSREPO}{\OSTree{}}1 {TAG}{\OSTag{}}1 {ARCH}{\arch{}}1 {-}{-}1]
[sms](*\#*) dnf install http://repos.openhpc.community/OpenHPC/VER/OSREPO/ARCH/ohpc-release-VER-1.TAG.ARCH.rpm
\end{lstlisting}
\else
% non-CentOS
\begin{lstlisting}[language=bash,keywords={},basicstyle=\fontencoding{T1}\fontsize{7.8}{10}\ttfamily,
	literate={VER}{\OHPCVerTree{}}1 {OSREPO}{\OSTree{}}1 {TAG}{\OSTag{}}1 {ARCH}{\arch{}}1 {-}{-}1]
[sms](*\#*) rpm -ivh http://repos.openhpc.community/OpenHPC/VER/OSREPO/ARCH/ohpc-release-VER-1.TAG.ARCH.rpm
\end{lstlisting}
\fi

\begin{center}
\begin{tcolorbox}[]
\small Many sites may find it useful or necessary to maintain a local copy of the
\OHPC{} repositories. To facilitate this need, standalone tar
archives are provided -- one containing a repository of binary packages as well as any
available updates, and one containing a repository of source RPMS. The tar files
also contain a simple bash script to configure the package manager to use the
local repository after download. To use, simply unpack the tarball where you
would like to host the local repository and execute the \texttt{make\_repo.sh} script.
Tar files for this release can be found at \href{http://repos.openhpc.community/dist/\OHPCVersion}
        {\color{blue}{http://repos.openhpc.community/dist/\OHPCVersion}}
\end{tcolorbox}
\end{center}

\subsection{Enable \xCAT{} repository for local use} \label{sec:enable_xcat}
\iftoggleverb{isxCATstateful}
To begin,
\else
Next,
\fi
enable use of the public \xCAT{} repository by adding it to the local list
of available package repositories. This also requires network access from
your {\em master} server to the internet, or alternatively, that
the repository be mirrored locally. In this case, we use the network.

% begin_ohpc_run
% ohpc_validation_newline
% ohpc_comment_header Enable xCAT repositories \ref{sec:enable_xcat}
\begin{lstlisting}[language=bash,keywords={},basicstyle=\fontencoding{T1}\fontsize{8.0}{10}\ttfamily,
	literate={VER}{\OHPCVerTree{}}1 {OSREPO}{\OSTree{}}1 {TAG}{\OSTag{}}1 {ARCH}{\arch{}}1 {-}{-}1]
[sms](*\#*) (*\install*) dnf-plugins-core
[sms](*\#*) (*\addrepo*) https://xcat.org/files/xcat/repos/yum/latest/xcat-core/xcat-core.repo
\end{lstlisting}
% end_ohpc_run



\noindent \xCAT{} has a number of dependencies that are required for
installation that are housed in separate public repositories for various
distributions. To enable additional repositories for local use, issue the following:

% begin_ohpc_run
\begin{lstlisting}[language=bash,keywords={},basicstyle=\fontencoding{T1}\fontsize{8.0}{10}\ttfamily,literate={ARCH}{\arch{}}1 {-}{-}1]
[sms](*\#*) (*\install*) centos-release-stream
[sms](*\#*) (*\addrepo*) http://xcat.org/files/xcat/repos/yum/devel/xcat-dep/rh8/ARCH/xcat-dep.repo
\end{lstlisting}
% end_ohpc_run

In addition to the \OHPC{}
\iftoggle{isxCAT}{and \xCAT{} package repositories,}{package repository,}
the {\em master} host also requires access to the standard base OS distro
repositories in order to resolve necessary dependencies. For \baseOS{}, the
requirements are to have access to the BaseOS, Appstream, Extras, CRB,
and EPEL repositories for which mirrors are freely available online:

\begin{itemize*}
\item Rocky-9
  (e.g. \href{http://download.rockylinux.org/pub/rocky/9/}
             {\color{blue}{http://download.rockylinux.org/pub/rocky/9/}} )
\item EPEL 9 (e.g. \href{http://download.fedoraproject.org/pub/epel/9/}
                        {\color{blue}{http://download.fedoraproject.org/pub/epel/9/}} )
\end{itemize*}

\noindent The public EPEL repository will be enabled automatically upon
installation of the \texttt{ohpc-release} package. Note that this does depend
on the Rocky Extras repository, which is shipped with Rocky and is typically
enabled by default.  In contrast, the CRB repository is typically
disabled in a standard install, but can be enabled from EPEL as follows:

\begin{lstlisting}[language=bash,literate={-}{-}1,keywords={},upquote=true]
[sms](*\#*) dnf install dnf-plugins-core
[sms](*\#*) dnf config-manager --set-enabled crb
\end{lstlisting}

\subsection{Installation template}
The collection of command-line instructions that follow in this guide, when
combined with local site inputs, can be used to implement a
bare-metal system installation and configuration. The format of these commands
is intended to be usable via direct cut and paste (with variable substitution
for site-specific settings). Alternatively, the \OHPC{} documentation package
(\texttt{docs-ohpc}) includes a template script which includes a summary of all
of the commands used herein. This script can be used in conjunction with a
simple text file to define the local site variables defined in the previous
section (\S~\ref{sec:inputs}) and is provided as a convenience for
administrators. For additional information on accessing this script, please see
Appendix~\ref{appendix:template_script}.




\subsection{Add provisioning services on {\em master} node} \label{sec:add_provisioning}
% -*- mode: latex; fill-column: 120; -*-

With the \OHPC{} and \xCAT{} repositories enabled, we can now begin adding desired components onto the {\em master}
server. This repository provides a number of aliases that group logical components together in order to help aid in this
process. For reference, a complete list of available group aliases and RPM packages available via \OHPC{} are provided
in Appendix~\ref{appendix:manifest}. To add support for provisioning services, the following commands illustrate
addition of a common base package followed by the \xCAT{} provisioning system.

% begin_ohpc_run
% ohpc_comment_header Add baseline OpenHPC and provisioning services \ref{sec:add_provisioning}
\begin{lstlisting}[language=bash,keywords={}]
# Install base meta-package
[sms](*\#*) (*\install*) ohpc-base

[sms](*\#*) (*\install*) xCAT

# enable xCAT tools for use in current shell
[sms](*\#*) . /etc/profile.d/xcat.sh
\end{lstlisting}
% ohpc_validation_newline
% end_ohpc_run



\input{common/enable_pxe}
HPC systems rely on synchronized clocks throughout the system and the
NTP protocol can be used to facilitate this synchronization. To enable NTP
services on the SMS host with a specific server \texttt{\$\{ntp\_server\}}, and
allow this server to serve as a local time server for the cluster,
issue the following:

% begin_ohpc_run
% ohpc_validation_comment Enable NTP services on SMS host
\begin{lstlisting}[language=bash,literate={-}{-}1,keywords={},upquote=true,keepspaces]
[sms](*\#*) systemctl enable chronyd.service
[sms](*\#*) echo "local stratum 10" >> /etc/chrony.conf
[sms](*\#*) echo "server ${ntp_server}" >> /etc/chrony.conf
[sms](*\#*) echo "allow all" >> /etc/chrony.conf
[sms](*\#*) systemctl restart chronyd
\end{lstlisting}
% end_ohpc_run

\begin{center}
\begin{tcolorbox}[]
\small Note that the ``allow all'' option specified for the chrony time daemon
allows all servers on the local network to be able to synchronize with the SMS
host. Alternatively, you can restrict access to fixed IP ranges and an example
config line allowing access to a local class B subnet is as follows:
\begin{lstlisting}[language=bash]
allow 192.168.0.0/16
\end{lstlisting}
\end{tcolorbox}
\end{center}


\subsection{Add resource management services on {\em master} node} \label{sec:add_rm}
\OHPC{} provides multiple options for distributed resource management. 
The following command adds the \SLURM{} workload manager server components to the
chosen {\em master} host. Note that client-side components will be added to
the corresponding compute image in a subsequent step.

% begin_ohpc_run
% ohpc_comment_header Add resource management services on master node \ref{sec:add_rm}
\begin{lstlisting}[language=bash,keywords={}]
# Install slurm server meta-package
[sms](*\#*) (*\install*) ohpc-slurm-server

# Use ohpc-provided file for starting SLURM configuration
[sms](*\#*) cp /etc/slurm/slurm.conf.ohpc /etc/slurm/slurm.conf
# Setup default cgroups file
[sms](*\#*) cp /etc/slurm/cgroup.conf.example /etc/slurm/cgroup.conf

# Identify resource manager hostname on master host
[sms](*\#*) perl -pi -e "s/SlurmctldHost=\S+/SlurmctldHost=${sms_name}/" /etc/slurm/slurm.conf
\end{lstlisting}
% end_ohpc_run

There are a wide variety of configuration options and plugins available
for \SLURM{} and the example config file illustrated above targets a fairly
basic installation. In particular, job completion data will be stored in a text
file (\texttt{/var/log/slurm\_jobcomp.log)} that can be used to log simple
accounting information. Sites who desire more detailed information, or want to
aggregate accounting data from multiple clusters, will likely want to enable the
database accounting back-end.  This requires a number of additional local modifications
(on top of installing \texttt{slurm-slurmdbd-ohpc}), and users are advised to
consult the online \href{https://slurm.schedmd.com/accounting.html}{\color{blue}{documentation}}
for more detailed information on setting up a database configuration for \SLURM{}.

\begin{center}
\begin{tcolorbox}[]
  \small SLURM requires enumeration of the physical hardware characteristics for
  compute nodes under its control. In particular, three configuration parameters
  combine to define consumable compute resources: {\em Sockets}, {\em
  CoresPerSocket}, and {\em ThreadsPerCore}. The default configuration file
  provided via \OHPC{} assumes the nodes are named c1-c4 and are dual-socket, 8
  cores per socket, and two threads per core for this 4-node example. If this
  does not reflect your local hardware, please update the configuration file at
  \path{/etc/slurm/slurm.conf} accordingly to match your nodes names and
  particular hardware. Be sure to run \texttt{scontrol reconfigure} to notify
  SLURM of the changes. Note that the SLURM project provides an easy-to-use
  online configuration tool that can be accessed
  \href{https://slurm.schedmd.com/configurator.html}{\color{blue} here}.
\end{tcolorbox}
\end{center}

% begin_ohpc_run
% ohpc_comment_header Update node configuration for slurm.conf
% ohpc_command if [[ ${update_slurm_nodeconfig} -eq 1 ]];then
% ohpc_indent 5
% ohpc_command perl -pi -e "s/^NodeName=.+$/#/" /etc/slurm/slurm.conf
% ohpc_command perl -pi -e "s/ Nodes=c\S+ / Nodes=${compute_prefix}[1-${num_computes}] /" /etc/slurm/slurm.conf
% ohpc_command echo -e ${slurm_node_config} >> /etc/slurm/slurm.conf
% ohpc_indent 0
% ohpc_command fi
% end_ohpc_run

Other versions of this guide are available that describe installation of alternate
resource management systems, and they can be found in the \texttt{docs-ohpc}
package.



\subsection{Optionally add \InfiniBand{} support services on {\em master} node} \label{sec:add_ofed}
\input{common/ibsupport_sms_centos}

\subsection{Optionally add \OmniPath{} support services on {\em master} node} \label{sec:add_opa}
\input{common/opasupport_sms_centos}

%\vspace*{-0.25cm}
\subsection{Complete basic \xCAT{} setup for {\em master} node} \label{sec:setup_xcat}
At this point, all of the packages necessary to use \xCAT{} on the {\em master}
host should be installed. Next, we enable support for local provisioning using
a second private interface (refer to Figure~\ref{fig:physical_arch}) and
register this network interface with \xCAT{}.

% begin_ohpc_run
% ohpc_comment_header Complete basic xCAT setup for master node \ref{sec:setup_xcat}
%\begin{verbatim}
\begin{lstlisting}[language=bash,literate={-}{-}1,keywords={},upquote=true,keepspaces]
# Enable internal interface for provisioning
[sms](*\#*) ip link set dev ${sms_eth_internal} up
[sms](*\#*) ip address add ${sms_ip}/${internal_netmask} broadcast + dev ${sms_eth_internal}

# Register internal provisioning interface with xCAT for DHCP
[sms](*\#*) chdef -t site dhcpinterfaces="xcatmn|${sms_eth_internal}"

\end{lstlisting}
%\end{verbatim}
% end_ohpc_run


\noindent \xCAT{} requires a network domain name specification for system-wide name
resolution. This value can be set to match your local DNS schema or given a
unique identifier such as ``local''. In this recipe, we leverage the
\texttt{\$domain\_name} variable to define as follows:

% begin_ohpc_run
% ohpc_validation_newline
% ohpc_validation_comment Define local domainname
\begin{lstlisting}[language=bash,keywords={},upquote=true,basicstyle=\footnotesize\ttfamily,literate={BOSVER}{\baseos{}}1]
[sms](*\#*) chdef -t site domain=${domain_name}
\end{lstlisting}


\subsection{Define {\em compute} image for provisioning}
% -*- mode: latex; fill-column: 120; -*-

With the provisioning services enabled, the next step is to define
\nottoggle{isxCATstateful}{and customize}
a system image that can subsequently be
used to provision one or more {\em compute} nodes. The following subsections highlight this process.

\subsubsection{Build initial BOS image} \label{sec:assemble_bos}
The following steps illustrate the process to build a minimal, default image for use with \xCAT{}. To begin, you will
first need to have a local copy of the ISO image available for the underlying OS. In this recipe, the relevant ISO image
is \texttt{Rocky-8.4-x86\_64-dvd1.iso} (available from the Rocky
\href{https://rockylinux.org/download/}{\color{blue}download} page).
We initialize the image
creation process using the \texttt{copycds} command assuming that the necessary ISO image is available locally in
\texttt{\$\{iso\_path\}} as follows:

% begin_ohpc_run
% ohpc_comment_header Initialize OS images for use with xCAT \ref{sec:assemble_bos}
\begin{lstlisting}[language=bash,literate={-}{-}1,keywords={},upquote=true,keepspaces,literate={BOSVER}{\baseos{}}1]
[sms](*\#*) copycds ${iso_path}/Rocky-8.4-x86_64-dvd1.iso
\end{lstlisting}
% end_ohpc_run

\noindent Once completed, several OS images should be available for use within \xCAT{}. These can be queried via:

\begin{lstlisting}[language=bash,literate={-}{-}1,keywords={},upquote=true,keepspaces,literate={BOSVER}{\baseos{}}1]
# Query available images
[sms](*\#*) lsdef -t osimage
centos8-x86_64-install-compute  (osimage)
centos8-x86_64-netboot-compute  (osimage)
centos8-x86_64-stateful-mgmtnode  (osimage)
centos8-x86_64-statelite-compute  (osimage)
\end{lstlisting}

In this example, we leverage
\iftoggleverb{isxCATstateful}
the stateful ({\em install}) image for provisioning of {\em
compute} nodes.  The other images, {\em
netboot-compute} and {\em statelite-compute}, are for stateless and statelite
installations, respectively. \OHPC{} provides a stateless \xCAT{} recipe,
follow that guide if you are interested in a stateless install.
\else
the stateless ({\em netboot}) image for compute nodes and proceed by using
\texttt{genimage} to initialize a chroot-based install. Note that the previous query highlights the existence of other
provisioning images as well. \OHPC{} provides a stateful \xCAT{} recipe,
follow that guide if you are interested in  stateful install.
%The \texttt{CHROOT} environment variable highlights the path and is used by
%subsequent commands to augment the basic installation.
\fi
Statelite installation is an intermediate type of install, in which a limited
number of files and directories persist across reboots.  Please consult
available xCAT \href{https://xcat-docs.readthedocs.io/en/stable/}{\color{blue}
documentation} if interested in this type of install.

\input{common/xcat_mkchroot_centos}
\vspace*{0.2cm}
\subsubsection{Add \OHPC{} components} \label{sec:add_components}
% -*- mode: latex; fill-column: 120; -*-

The \texttt{genimage} process used in the previous step is designed to provide a minimal \baseOS{} configuration. Next,
we add additional components to include resource management client services, \InfiniBand{} drivers, and other additional
packages to support the default \OHPC{} environment. This process augments the chroot-based install performed by
\texttt{genimage} to modify the base provisioning image and requires access the BOS and \OHPC{} repositories to resolve
package install requests. The following steps are used to first enable the necessary package repositories for use within
the chroot.

% begin_ohpc_run
% ohpc_comment_header Enable repos in chroot \ref{sec:add_components}
\begin{lstlisting}[language=bash,literate={-}{-}1,keywords={},upquote=true]
[sms](*\#*) cp /etc/yum.repos.d/OpenHPC.repo $CHROOT/etc/yum.repos.d
[sms](*\#*) cp /etc/yum.repos.d/epel.repo $CHROOT/etc/yum.repos.d
\end{lstlisting}
% end_ohpc_run

\noindent Next, install a base compute package and disable firewall:
% begin_ohpc_run
% ohpc_comment_header Add OpenHPC base components to compute image \ref{sec:add_components}
\begin{lstlisting}[language=bash,literate={-}{-}1,keywords={},upquote=true]
# Install compute node base meta-package
[sms](*\#*) (*\chrootinstall*) ohpc-base-compute
# Disable firewall for computes
[sms](*\#*) chroot $CHROOT systemctl disable firewalld
\end{lstlisting}
% end_ohpc_run

\noindent Now, we can include additional components to the compute instance using
\texttt{\pkgmgr{}} to install into the chroot location defined previously:


%\newpage
% begin_ohpc_run
% ohpc_validation_comment Add OpenHPC components to compute instance
\begin{lstlisting}[language=bash,literate={-}{-}1,keywords={},upquote=true,
    literate={BOSSHORT}{\baseosshort{}}1 {ARCH}{\arch{}}1]
# copy credential files into $CHROOT to ensure consistent uid/gids for slurm/munge at
# install. Note that these will be synchronized with future updates via the provisioning system.
[sms](*\#*) cp /etc/passwd /etc/group  $CHROOT/etc
# Add Slurm client support meta-package
[sms](*\#*) (*\chrootinstall*) ohpc-slurm-client
# Register Slurm server with computes (using "configless" option)
[sms](*\#*) echo SLURMD_OPTIONS="--conf-server ${sms_ip}" > $CHROOT/etc/sysconfig/slurmd
# Add Network Time Protocol (NTP) support
[sms](*\#*) (*\chrootinstall*) chrony
# Identify master host as local NTP server
[sms](*\#*) echo "server ${sms_ip} iburst" >> $CHROOT/etc/chrony.conf
# Add kernel
[sms](*\#*) (*\chrootinstall*) kernel
# Include matching kernel from head node into compute image
[sms](*\#*) genimage BOSSHORT-ARCH-netboot-compute -k `uname -r`
# Re-enable BaseOS repo
[sms](*\#*) dnf config-manager --installroot=$CHROOT --enable BaseOS
# Include modules user environment
[sms](*\#*) (*\chrootinstall*)  --enablerepo=powertools lmod-ohpc
\end{lstlisting}

% ohpc_comment_header Optionally add InfiniBand support services in compute node image \ref{sec:add_components}
% ohpc_command if [[ ${enable_ib} -eq 1 ]];then
% ohpc_indent 5
\begin{lstlisting}[language=bash,literate={-}{-}1,keywords={},upquote=true]
# Optionally add IB support and enable
[sms](*\#*) (*\groupchrootinstall*) "InfiniBand Support"
\end{lstlisting}
% ohpc_indent 0
% ohpc_command fi
% end_ohpc_run

%\vspace*{0.5cm}
\subsubsection{Customize system configuration} \label{sec:master_customization}
\input{common/xcat_chroot_customize_centos}

% Additional commands when additional computes are requested

% begin_ohpc_run
% ohpc_validation_newline
% ohpc_validation_comment Update basic slurm configuration if additional computes defined
% ohpc_command if [ ${num_computes} -gt 4 ];then
% ohpc_command    perl -pi -e "s/^NodeName=(\S+)/NodeName=${compute_prefix}[1-${num_computes}]/" /etc/slurm/slurm.conf
% ohpc_command    perl -pi -e "s/^PartitionName=normal Nodes=(\S+)/PartitionName=normal Nodes=${compute_prefix}[1-${num_computes}]/" /etc/slurm/slurm.conf

% ohpc_command    perl -pi -e "s/^NodeName=(\S+)/NodeName=${compute_prefix}[1-${num_computes}]/" $CHROOT/etc/slurm/slurm.conf
% ohpc_command    perl -pi -e "s/^PartitionName=normal Nodes=(\S+)/PartitionName=normal Nodes=${compute_prefix}[1-${num_computes}]/" $CHROOT/etc/slurm/slurm.conf
% ohpc_command fi
% end_ohpc_run

%\clearpage
\subsubsection{Additional Customization ({\em optional})} \label{sec:addl_customizations}
This section highlights common additional customizations that can {\em
optionally} be applied to the local cluster environment. These customizations
include:

\begin{multicols}{2}
\begin{itemize*}
\iftoggleverb{isx86}
\item Add InfiniBand or Omni-Path drivers
\item Increase memlock limits
\fi

\nottoggle{ispbs}{\item Restrict ssh access to compute resources}

\iftoggleverb{isx86}
\item Add \beegfs{} client
\item Add \Lustre{} client
\fi

\iftoggle{isWarewulf}{\item Enable syslog forwarding}

\item Add \clustershell{}
\item Add \mrsh{}
\item Add \genders{}
\item Add \conman{}
\iftoggleverb{isSLURM}
\item Add \GEOPM{}
\fi
\end{itemize*}
\end{multicols}

\noindent Details on the steps required for each of these customizations are
discussed further in the following sections.


\vspace*{0.3cm}
\paragraph{Increase locked memory limits}
In order to utilize \InfiniBand{} or Omni-Path as the underlying high speed interconnect, it is
generally necessary to increase the locked memory settings for system
users. This can be accomplished by updating
the \texttt{/etc/security/limits.conf} file and this should be performed within
the {{\em compute} image and on all job submission hosts. In this recipe, jobs
are submitted from the {\em master} host, and the following commands can be
used to update the maximum locked memory settings on both the master host and
the compute image:

% begin_ohpc_run
% ohpc_validation_newline
% ohpc_validation_comment Update memlock settings
\begin{lstlisting}[language=bash,keywords={},upquote=true]
# Update memlock settings on master
[sms](*\#*) perl -pi -e 's/# End of file/\* soft memlock unlimited\n$&/s' /etc/security/limits.conf
[sms](*\#*) perl -pi -e 's/# End of file/\* hard memlock unlimited\n$&/s' /etc/security/limits.conf

# Update memlock settings within compute image
[sms](*\#*) perl -pi -e 's/# End of file/\* soft memlock unlimited\n$&/s' $CHROOT/etc/security/limits.conf
[sms](*\#*) perl -pi -e 's/# End of file/\* hard memlock unlimited\n$&/s' $CHROOT/etc/security/limits.conf
\end{lstlisting}
% end_ohpc_run




\vspace*{0.3cm}
\paragraph{Enable ssh control via resource manager}
\input{common/slurm_pam}

\vspace*{0.5cm}
\paragraph{Add \Lustre{} client} \label{sec:lustre_client}
To add \Lustre{} client support on the cluster, it necessary to install the client
and associated modules on each host needing to access a \Lustre{} file system. In
this recipe, it is assumed that the \Lustre{} file system is hosted by servers
that are pre-existing and are not part of the install process. Outlining the
variety of \Lustre{} client mounting options is beyond the scope of this document,
%(please consult \Lustre{} documentation for more details on failover configuration
%support and networking options),
but the general requirement is to add a mount entry for the desired file system
that defines the management server (MGS) and underlying network transport
protocol. To add client mounts on both the {\em master} server and {\em
compute} image, the following commands can be used. Note that the \Lustre{} file
system to be mounted is identified by the \texttt{\$\{mgs\_fs\_name\}} variable.
In this example, the file system is configured to be mounted locally
as \path{/mnt/lustre}.

\vspace*{0.3cm}
% begin_ohpc_run
% ohpc_validation_newline
% ohpc_validation_comment Enable Optional packages
% ohpc_validation_newline
% ohpc_command if [[ ${enable_lustre_client} -eq 1 ]];then
% ohpc_indent 5

% ohpc_validation_comment Install Lustre client on master
\begin{lstlisting}[language=bash,keywords={},upquote=true]
# Add Lustre client software to master host
[sms](*\#*) (*\install*) lustre-client-ohpc
\end{lstlisting}
% end_ohpc_run

% begin_ohpc_run
% ohpc_validation_comment Enable lustre in WW compute image
\begin{lstlisting}[language=bash,keywords={},upquote=true]
# Include Lustre client software in compute image
[sms](*\#*) (*\chrootinstall*) lustre-client-ohpc

# Include mount point and file system mount in compute image
[sms](*\#*) mkdir $CHROOT/mnt/lustre
[sms](*\#*) echo "${mgs_fs_name} /mnt/lustre lustre defaults,localflock,noauto,x-systemd.automount 0 0" \
    >> $CHROOT/etc/fstab
\end{lstlisting}
% end_ohpc_run

\input{common/lustre-client-post}

\paragraph{Add \clustershell{}}
\input{common/clustershell}

%\clearpage
\paragraph{Add \genders{}}
\input{common/genders}

\paragraph{Add Magpie}
\input{common/magpie}

\paragraph{Add \conman{}} \label{sec:add_conman}
\conman{} is a serial console management program designed to support a large
number of console devices and simultaneous users. It supports logging console
device output and connecting to compute node consoles via IPMI
serial-over-lan. Installation and example configuration is outlined below.

% begin_ohpc_run
% ohpc_validation_newline
% ohpc_validation_comment Optionally, enable conman and configure
% ohpc_command if [[ ${enable_ipmisol} -eq 1 ]];then
% ohpc_indent 5
\begin{lstlisting}[language=bash,keywords={},upquote=true]
# Install conman to provide a front-end to compute consoles and log output
[sms](*\#*) (*\install*) conman-ohpc

# Configure conman for computes (note your IPMI password is required for console access)
[sms](*\#*) for ((i=0; i<$num_computes; i++)) ; do
              echo -n 'CONSOLE name="'${c_name[$i]}'" dev="ipmi:'${c_bmc[$i]}'" '
              echo 'ipmiopts="'U:${bmc_username},P:${IPMI_PASSWORD:-undefined},W:solpayloadsize'"'
        done >> /etc/conman.conf

# Enable and start conman
[sms](*\#*) systemctl enable conman
[sms](*\#*) systemctl start conman
\end{lstlisting}
% ohpc_indent 0
% ohpc_command fi
% end_ohpc_run

\iftoggleverb{isWarewulf}
\noindent Note that an additional kernel boot option is typically necessary to
enable serial console output. This option is highlighted in \S\ref{sec:optional_kargs} after
compute nodes have been registered with the provisioning system.
\fi

\iftoggleverb{isxCAT}
\noindent Note that additional options are typically necessary to
enable serial console output. These are setup during the node registration
process in \S\ref{sec:xcat_add_nodes}
\fi




\paragraph{Add \nhc{}} \label{sec:add_nhc}
Resource managers often provide for a periodic "node health check" to be
performed on each compute node to verify that the node is working
properly. Nodes which are determined to be "unhealthy" can be marked as down or
offline so as to prevent jobs from being scheduled or run on them. This helps
increase the reliability and throughput of a cluster by reducing preventable
job failures due to misconfiguration, hardware failure, etc. OpenHPC
distributes \nhc{} to fulfill this requirement.

In a typical scenario, the \nhc{} driver script is run periodically on each compute
node by the resource manager client daemon. It loads its
configuration file to determine which checks are to be run on the current node
(based on its hostname). Each matching check is run, and if a failure is
encountered, \nhc{} will exit with an error message describing the problem. It can
also be configured to mark nodes offline so that the scheduler will not assign
jobs to bad nodes, reducing the risk of system-induced job failures.

% begin_ohpc_run
% ohpc_validation_newline
% ohpc_validation_comment Optionally, enable nhc and configure
\begin{lstlisting}[language=bash,keywords={},upquote=true]
# Install NHC on master and compute nodes
[sms](*\#*) (*\install*) nhc-ohpc
[sms](*\#*) (*\chrootinstall*) nhc-ohpc
\end{lstlisting}
% end_ohpc_run


\input{common/nhc_slurm}

\subsubsection{Identify files for synchronization} \label{sec:file_import}
\input{common/import_xcat_files}
\noindent Similarly, to import the
%global Slurm configuration file and the
cryptographic
key
%and associated file permissions
that is required by the {\em munge}
authentication library to be available on every host in the resource management
pool, issue the following:

% begin_ohpc_run
\begin{lstlisting}[language=bash,literate={-}{-}1,keywords={},upquote=true]
[sms](*\#*) echo "/etc/munge/munge.key -> /etc/munge/munge.key" >>/install/custom/netboot/compute.synclist
\end{lstlisting}
% \end_ohpc_run

\begin{center}
\begin{tcolorbox}[]
\small
The ``\texttt{updatenode compute -F}'' command can be used to distribute changes made to any
defined synchronization files on the SMS host. Users wishing to automate this process may
want to consider adding a crontab entry to perform this action at defined intervals.
\end{tcolorbox}
\end{center}

\input{common/finalize_xcat_provisioning}

\vspace*{0.9cm}
\subsection{Add compute nodes into \xCAT{} database} \label{sec:xcat_add_nodes}
%\subsubsection{Register nodes for provisioning}

\noindent Next, we add {\em compute} nodes and define their properties as
objects in \xCAT{} database.
These hosts are grouped logically into a group named {\em
compute} to facilitate group-level commands used later in the recipe.  Note the
use of variable names for the desired compute hostnames, node IPs, MAC
addresses, and BMC login credentials, which should be modified to accommodate
local settings and hardware. To enable serial console access via  \xCAT{},
{\texttt serialport} and {\texttt serialspeed}
properties are also defined.

% begin_ohpc_run
% ohpc_validation_newline
% ohpc_validation_comment Add hosts to cluster \ref{sec:xcat_add_nodes}
\begin{lstlisting}[language=bash,keywords={},upquote=true,basicstyle=\footnotesize\ttfamily,]
# Define nodes as objects in xCAT database
[sms](*\#*) for ((i=0; i<$num_computes; i++)) ; do
		mkdef -t node ${c_name[$i]} groups=compute,all ip=${c_ip[$i]} mac=${c_mac[$i]} netboot=xnba \
		arch=x86_64 bmc=${c_bmc[$i]} bmcusername=${bmc_username} bmcpassword=${bmc_password} \
		mgt=ipmi serialport=0 serialspeed=115200
        done
\end{lstlisting}
% end_ohpc_run

\begin{center}
  \begin{tcolorbox}[]
    \small
Defining nodes one-by-one, as done above, is only efficient
for a small number of nodes. For larger node counts,
\xCAT{} provides capabilities for automated detection and
configuration.
Consult the
\href{http://xcat-docs.readthedocs.io/en/stable/guides/admin-guides/manage_clusters/ppc64le/discovery/}{\color{blue}\xCAT{}
Hardware Discovery \& Define Node Guide}. Alternatively, confluent, a tool
related to \xCAT{}, also has robust discovery capabilities and can be used to
\href{https://hpc.lenovo.com/users/documentation/confluentdisco.html}{\color{blue}detect
and auto-configure} compute hosts.
\end{tcolorbox}
\end{center}

\iftoggleverb{isxCATstateful}
Next, we set root password for {\em compute} nodes. Note that \xCAT{} sets up
\texttt{ssh} keys to enable password-less logins for root across the cluster,
but the password is required for kickstarting the stateful install.

% begin_ohpc_run
% ohpc_validation_comment Set default root password
\begin{lstlisting}[language=bash,keywords={},upquote=true,basicstyle=\footnotesize\ttfamily,]
[sms](*\#*) chtab key=system passwd.username=root passwd.password=`openssl rand -base64 12`
\end{lstlisting}
% end_ohpc_run
\fi

%\clearpage
If enabling {\em optional} IPoIB functionality (e.g. to support Lustre over \InfiniBand{}), additional
settings are required to define the IPoIB network with \xCAT{} and specify
desired IP settings for each compute. This can be accomplished as follows for
the {\em ib0} interface:

% begin_ohpc_run
% ohpc_validation_newline
% ohpc_validation_comment Setup IPoIB networking
% ohpc_command if [[ ${enable_ipoib} -eq 1 ]];then
% ohpc_indent 5
\begin{lstlisting}[language=bash,keywords={},upquote=true,basicstyle=\footnotesize\ttfamily]
# Define ib0 netmask
[sms](*\#*) chdef -t network -o ib0 mask=$ipoib_netmask net=${c_ipoib[0]}

# Enable secondary NIC configuration
[sms](*\#*) chdef compute -p postbootscripts=confignics

# Register desired IPoIB IPs per compute
[sms](*\#*) for ((i=0; i<$num_computes; i++)) ; do
		chdef ${c_name[i]} nicips.ib0=${c_ipoib[i]} nictypes.ib0="InfiniBand" nicnetworks.ib0=ib0
        done
\end{lstlisting}
% ohpc_indent 0
% ohpc_command fi
% end_ohpc_run

%\clearpage
With the desired compute nodes and domain identified, the remaining steps in the
provisioning configuration process are to define the provisioning mode and
image for the {\em compute} group and use \xCAT{} commands to complete
configuration for network services like DNS and DHCP. These tasks are
accomplished as follows:

%\clearpage
% begin_ohpc_run
% ohpc_validation_newline
% ohpc_validation_comment Complete networking setup, associate provisioning image
\begin{lstlisting}[language=bash,keywords={},upquote=true,basicstyle=\footnotesize\ttfamily,literate={BOSSHORT}{\baseosshort{}}1 {IMAGE}{\installimage{}}1]
# Complete network service configurations
[sms](*\#*) makehosts
[sms](*\#*) makenetworks
[sms](*\#*) makedhcp -n
[sms](*\#*) makedns -n

# Associate desired provisioning image for computes
[sms](*\#*) nodeset compute osimage=BOSSHORT-x86_64-IMAGE-compute
\end{lstlisting}

%%% If the Lustre client was enabled for computes in \S\ref{sec:lustre_client}, you
%%% should be able to mount the file system post-boot using the fstab entry
%%% (e.g. via ``\texttt{mount /mnt/lustre}''). Alternatively, if
%%% you prefer to have the file system mounted automatically at boot time, a simple
%%% postscript can be created and registered with \xCAT{} for this purpose as follows.
%%%
%%% % begin_ohpc_run
%%% % ohpc_validation_newline
%%% % ohpc_validation_comment Optionally create xCAT postscript to mount Lustre client
%%% % ohpc_command if [ ${enable_lustre_client} -eq 1 ];then
%%% % ohpc_indent 5
%%% \begin{lstlisting}[language=bash,keywords={},upquote=true,basicstyle=\footnotesize\ttfamily,literate={BOSVER}{\baseos{}}1]
%%% # Optionally create postscript to mount Lustre client at boot
%%% [sms](*\#*) echo '#!/bin/bash' > /install/postscripts/lustre-client
%%% [sms](*\#*) echo 'mount /mnt/lustre' >> /install/postscripts/lustre-client
%%% [sms](*\#*) chmod 755 /install/postscripts/lustre-client
%%% # Register script for computes
%%% [sms](*\#*) chdef compute -p postscripts=lustre-client
%%% \end{lstlisting}
%%% % ohpc_indent 0
%%% % ohpc_command fi
%%% % end_ohpc_run
%%%



\vspace*{0.1cm}
\subsection{Boot compute nodes} \label{sec:boot_computes}
Prior to booting the {\em compute} hosts, we configure them to use PXE as their
next boot mode. After the initial PXE, ensuing boots will return to using the default boot device
specified in the BIOS.

% begin_ohpc_run
% ohpc_comment_header Set nodes to netboot \ref{sec:boot_computes}
\begin{lstlisting}[language=bash,keywords={},upquote=true]
[sms](*\#*) rsetboot compute net
\end{lstlisting} 
% end_ohpc_run

At this point, the {\em master} server should be able to boot the newly defined
compute nodes. This is done by using the \texttt{rpower} \xCAT{} command
leveraging IPMI protocol set up during the the {\em compute} node definition
in \S~\ref{sec:xcat_add_nodes}. The following power cycles each of the
desired hosts.


% begin_ohpc_run
% ohpc_comment_header Boot compute nodes
\begin{lstlisting}[language=bash,keywords={},upquote=true]
[sms](*\#*) rpower compute reset
\end{lstlisting} 
% end_ohpc_run

Once kicked off, the boot process should take about \nottoggle{isxCATstateful}{5
minutes (depending on BIOS post times)}{5 to 10 minutes}.  You can monitor the
provisioning by using the \texttt{rcons} command, which displays serial console
for a selected node. Note that the escape sequence
is \texttt{CTRL-e c .} typed sequentially. 

Successful provisioning can be verified by a parallel command on the compute
nodes. The default install provides two such tools: \xCAT{}-provided
\texttt{psh} command, which uses \xCAT{} node names and groups,
and \texttt{pdsh}, which works independently.  For example, to run a command on
the newly imaged compute hosts using \texttt{psh}, execute the following:

\begin{lstlisting}[language=bash]
[sms](*\#*) psh compute uptime
c1:  12:56:50 up 14 min,  0 users,  load average: 0.00, 0.01, 0.04
c2:  12:56:50 up 13 min,  0 users,  load average: 0.00, 0.02, 0.05
c3:  12:56:50 up 14 min,  0 users,  load average: 0.00, 0.02, 0.05
c4:  12:56:50 up 14 min,  0 users,  load average: 0.00, 0.01, 0.04
\end{lstlisting}
Note that the equivalent \texttt{pdsh} command is \texttt{pdsh -w ${compute_prefix}[1-${num_computes}] uptime}. 


%\vspace*{-0.50cm}
\section{Install \OHPC{} Development Components}
\input{common/dev_intro.tex}

\vspace*{-0.15cm}
\subsection{Development Tools} \label{sec:install_dev_tools}
\input{common/dev_tools}

%\vspace*{-0.15cm}
\subsection{Compilers} \label{sec:install_compilers}
\OHPC{} presently packages the \GNU{} compiler toolchain integrated with the 
underlying Lmod modules system in a hierarchical fashion. The modules
system will conditionally present compiler-dependent software based on the
toolchain currently loaded. 

% begin_ohpc_run
% ohpc_comment_header Install Compilers \ref{sec:install_compilers}
\begin{lstlisting}[language=bash]
[sms](*\#*) (*\install*) gnu13-compilers-ohpc
\end{lstlisting}
% end_ohpc_run

%%% The llvm compiler toolchains are also provided as a standalone additional
%%% compiler family (ie. users can easily switch between gcc/clang environments),
%%% but we do not provide the full complement of downstream library builds.
%%% 
%%% % begin_ohpc_run
%%% % ohpc_comment_header Install llvm Compilers
%%% \begin{lstlisting}[language=bash]
%%% [sms](*\#*) (*\install*) llvm5-compilers-ohpc
%%% \end{lstlisting}
%%% % end_ohpc_run


\vspace{0.2cm}
\subsection{MPI Stacks} \label{sec:mpi}
For MPI development and runtime support, \OHPC{} provides pre-packaged builds
for a variety of MPI families and transport layers. Currently available options
and their applicability to various network transports are summarized in
Table~\ref{table:mpi}.  The command that follows installs a starting set of MPI
families that are compatible with both ethernet and high-speed fabrics.

\iftoggleverb{isx86}
% x86_64

\begin{table}[h]
\caption{Available MPI variants} \label{table:mpi}
\centering
\begin{tabular}{@{\hspace*{0.2cm}} *5l @{}}    \toprule
                                  & Ethernet (TCP)                 & \InfiniBand{}                  & \IntelR{} Omni-Path            \\ \midrule
\rowcolor{black!10} MPICH (ofi) & \multicolumn{1}{c}{\checkmark} & \multicolumn{1}{c}{\checkmark} & \multicolumn{1}{c}{\checkmark} \\
 MPICH (ucx)       & \multicolumn{1}{c}{\checkmark} & \multicolumn{1}{c}{\checkmark} & \multicolumn{1}{c}{\checkmark} \\
\rowcolor{black!10} MVAPICH2                          &                                & \multicolumn{1}{c}{\checkmark} &                                \\
MVAPICH2 (psm2) &                              &                                & \multicolumn{1}{c}{\checkmark} \\
\rowcolor{black!10} OpenMPI (ofi/ucx)            & \multicolumn{1}{c}{\checkmark} & \multicolumn{1}{c}{\checkmark} & \multicolumn{1}{c}{\checkmark} \\
%\rowcolor{black!10} OpenMPI (PMIx) & \multicolumn{1}{c}{\checkmark} & \multicolumn{1}{c}{\checkmark} & \multicolumn{1}{c}{\checkmark} \\ \bottomrule
\end{tabular}
\end{table}

\else
% aarch64

\begin{table}[h]
\caption{Available MPI builds} \label{table:mpi}
\centering
\begin{tabular}{@{\hspace*{0.2cm}} *5l @{}}    \toprule
                                  & Ethernet (TCP)                 & \InfiniBand{}                              \\ \midrule
\rowcolor{black!10} MPICH         & \multicolumn{1}{c}{\checkmark} &                                            \\
\rowcolor{black!10} OpenMPI                           & \multicolumn{1}{c}{\checkmark} & \multicolumn{1}{c}{\checkmark} \\
\end{tabular}
\end{table}

\fi

% begin_ohpc_run
% ohpc_comment_header Install MPI Stacks \ref{sec:mpi}
% ohpc_command if [[ ${enable_mpi_defaults} -eq 1 ]];then
% ohpc_indent 5
\begin{lstlisting}[language=bash]
[sms](*\#*) (*\install*) openmpi5-pmix-gnu13-ohpc mpich-ofi-gnu13-ohpc
\end{lstlisting}
% ohpc_indent 0
% ohpc_command fi
% end_ohpc_run

Note that OpenHPC 2.x introduces the use of two related transport layers for
the MPICH and OpenMPI builds that support a variety of underlying
fabrics: \href{https://www.openucx.org}{UCX} (Unified Communication X)
and \href{https://ofiwg.github.io/libfabric/}{OFI} (OpenFabrics interfaces).
In the case of OpenMPI, a monolithic build is provided which supports both
transports and end-users can customize their runtime preferences with
environment variables. For MPICH, two separate builds are provided and the
example above highlighted installing the {\texttt ofi} variant.  However, the
packaging is designed such that both versions can be installed simultaneously
and users can switch between the two via normal module command
semantics. Alternatively, a site can choose to install the {\texttt ucx} variant
instead as a drop-in MPICH replacement:

\begin{lstlisting}[language=bash]
[sms](*\#*) (*\install*) mpich-ucx-gnu13-ohpc
\end{lstlisting}

In the case where both MPICH variants are installed, two modules will be
visible in the end-user environment and an example of this configuration is
highlighted is below.

\begin{lstlisting}[language=bash]
[sms](*\#*) module avail mpich

-------------------- /opt/ohpc/pub/moduledeps/gnu13---------------------
   mpich/3.4.3-ofi    mpich/3.4.3-ucx (D)
\end{lstlisting}

If your system includes \InfiniBand{} and you enabled underlying support in
\S\ref{sec:add_ofed} and \S\ref{sec:addl_customizations}, an additional
MVAPICH2 family is available for use:

% begin_ohpc_run
% ohpc_validation_newline
% ohpc_command if [[ ${enable_ib} -eq 1 ]];then
% ohpc_indent 5
\begin{lstlisting}[language=bash]
[sms](*\#*) (*\install*) mvapich2-gnu13-ohpc
\end{lstlisting}
% ohpc_indent 0
% ohpc_command fi
% end_ohpc_run

Alternatively, if your system includes \IntelR{} \OmniPath{}, use the (\texttt{psm2})
variant of MVAPICH2 instead:

% begin_ohpc_run
% ohpc_command if [[ ${enable_opa} -eq 1 ]];then
% ohpc_indent 5
\begin{lstlisting}[language=bash]
[sms](*\#*) (*\install*) mvapich2-psm2-gnu13-ohpc
\end{lstlisting}
% ohpc_indent 0
% ohpc_command fi
% end_ohpc_run

%%--
%% https://github.com/openhpc/ohpc/issues/1273
%% disabling until we can get pmix/openmpi/slurm to play nicely
%%--
%% An additional OpenMPI build variant is listed in Table~\ref{table:mpi} which
%% enables \href{https://pmix.github.io/pmix/}{\color{blue}{PMIx}} job launch
%% support for use with \SLURM{}. This optional variant is
%% available as \texttt{openmpi5-pmix-slurm-gnu9-ohpc}.


\subsection{Performance Tools} \label{sec:install_perf_tools}
\OHPC{} provides a variety of open-source tools to aid in application 
performance analysis (refer to Appendix~\ref{appendix:manifest} for a listing
of available packages). This group of tools can be installed as follows:

% begin_ohpc_run
% ohpc_comment_header Install Performance Tools \ref{sec:install_perf_tools}
\begin{lstlisting}[language=bash,keywords={},literate={-}{-}1]
# Install perf-tools meta-package
[sms](*\#*) (*\install*) ohpc-gnu13-perf-tools
\end{lstlisting}
% end_ohpc_run


\subsection{Setup default development environment}
System users often find it convenient to have a default development environment
in place so that compilation can be performed directly for parallel programs
requiring MPI. This setup can be conveniently enabled via modules and the \OHPC{}
modules environment is pre-configured to load an \texttt{ohpc} module on login
(if present). The following package install provides a default
environment that enables autotools, the \GNU{} compiler toolchain, and the
OpenMPI stack.

% begin_ohpc_run
\begin{lstlisting}[language=bash]
[sms](*\#*) (*\install*) lmod-defaults-gnu13-openmpi4-ohpc
\end{lstlisting}
% end_ohpc_run

\begin{center}
\begin{tcolorbox}[]
\small
\iftoggleverb{isx86}
If you want to change the default environment from the suggestion above, \OHPC{}
also provides the \GNU{} compiler toolchain with the MPICH and MVAPICH2 stacks:
\fi

\iftoggleverb{isaarch}
If you want to change the default environment from the suggestion above, \OHPC{}
also provides additional default options using the \GNU{} compiler toolchain
with multiple MPICH variants or MVAPICH2. Relevant lmod-default packages names
are as follows:
\fi

\begin{itemize*}
\item lmod-defaults-gnu13-mpich-ofi-ohpc
\item lmod-defaults-gnu13-mpich-ucx-ohpc
\iftoggleverb{isx86}
\item lmod-defaults-gnu13-mvapich2-ohpc
\fi
\end{itemize*}
\end{tcolorbox}
\end{center}


\vspace*{0.2cm}
\subsection{3rd Party Libraries and Tools} \label{sec:3rdparty}
\input{common/third_party_libs_intro}

\begin{center}
\begin{tcolorbox}[]
\small
\OHPC{}-provided 3rd party builds are configured to be installed
into a common top-level repository so that they can be easily exported to
desired hosts within the cluster. This common top-level path
(\path{/opt/ohpc/pub}) was previously configured to be mounted on {\em
 compute} nodes in \S\ref{sec:master_customization}, so the packages will be
immediately available for use on the cluster after installation on the {\em
 master} host.
\end{tcolorbox}
\end{center}

%\iftoggle{isCentOS}{\clearpage}
%\nottoggle{isCentOS}{\clearpage}

For convenience, \OHPC{} provides package aliases for these 3rd party libraries
and utilities that can be used to install available libraries for use with the
GNU compiler family toolchain. For parallel libraries, aliases are grouped by
MPI family toolchain so that administrators can choose a subset should they
favor a particular MPI stack.  Please refer to Appendix~\ref{appendix:manifest}
for a more detailed listing of all available packages in each of these functional
areas. To install all available package offerings within \OHPC{}, issue the
following:

% begin_ohpc_run
% ohpc_comment_header Install 3rd Party Libraries and Tools \ref{sec:3rdparty}
\begin{lstlisting}[language=bash,keywords={},upquote=true,keepspaces]
# Install 3rd party libraries/tools meta-packages built with GNU toolchain
[sms](*\#*) (*\install*) ohpc-gnu13-serial-libs
[sms](*\#*) (*\install*) ohpc-gnu13-io-libs
[sms](*\#*) (*\install*) ohpc-gnu13-python-libs
[sms](*\#*) (*\install*) ohpc-gnu13-runtimes
\end{lstlisting}
% end_ohpc_run




\vspace*{0.1cm}

% begin_ohpc_run
% ohpc_command if [[ ${enable_mpi_defaults} -eq 1 ]];then
% ohpc_indent 5
\begin{lstlisting}[language=bash,keywords={},upquote=true,keepspaces]
# Install parallel lib meta-packages for all available MPI toolchains
[sms](*\#*) (*\install*) ohpc-gnu13-mpich-parallel-libs
[sms](*\#*) (*\install*) ohpc-gnu13-openmpi5-parallel-libs
\end{lstlisting}
% ohpc_indent 0
% ohpc_command fi
% ohpc_command if [[ ${enable_ib} -eq 1 ]];then
% ohpc_indent 5
% ohpc_command (*\install*) ohpc-gnu13-mvapich2-parallel-libs
% ohpc_indent 0
% ohpc_command fi
% ohpc_command if [[ ${enable_opa} -eq 1 ]];then
% ohpc_indent 5
% ohpc_command (*\install*) ohpc-gnu13-mvapich2-parallel-libs
% ohpc_indent 0
% ohpc_command fi
% end_ohpc_run


\subsection{Optional Development Tool Builds} \label{sec:3rdparty_intel}
In addition to the 3rd party development libraries built using the open source
toolchains mentioned in \S\ref{sec:3rdparty}, \OHPC{} also provides {\em
  optional} compatible builds for use with the compilers and MPI stack included
in newer versions of the \IntelR{}~oneAPI HPC Toolkit (using the {\em classic}
compiler variants).  These
packages provide a similar hierarchical user
environment experience as other compiler and MPI families present in \OHPC{}.

To take advantage of the available builds, \OHPC{} provides a convenience
package to enable the oneAPI repository locally along with compatibility
packages that integrate oneAPI-generated compiler and MPI modulefiles within
the standard \OHPC{} user environment. To enable the \IntelR{} oneAPI
repository and install minimum compiler and MPI requirements for \OHPC{}
packaging, issue the following:

% begin_ohpc_run
% ohpc_comment_header Install Intel oneAPI tools \ref{sec:3rdparty_intel}
% ohpc_command if [[ ${enable_intel_packages} -eq 1 ]];then
% ohpc_indent 5
\begin{lstlisting}[language=bash,keywords={},upquote=true,keepspaces]
# Enable Intel oneAPI and install OpenHPC compatibility packages
[sms](*\#*) (*\install*) intel-oneapi-toolkit-release-ohpc
[sms](*\#*) rpm --import https://yum.repos.intel.com/intel-gpg-keys/GPG-PUB-KEY-INTEL-SW-PRODUCTS.PUB
[sms](*\#*) (*\install*) intel-compilers-devel-ohpc
[sms](*\#*) (*\install*) intel-mpi-devel-ohpc
\end{lstlisting}


\begin{center}
\begin{tcolorbox}[]
As noted in \S\ref{sec:master_customization}, the default installation path for
\OHPC{} (\texttt{/opt/ohpc/pub}) is exported over NFS from the {\em master} to the 
compute nodes, but the \IntelR{} oneAPI HPC Toolkit packages install to a top-level path of 
\texttt{/opt/intel}. To make the \IntelR{} compilers available to the compute 
nodes one must add an additional NFS export
for \texttt{/opt/intel} that is mounted on desired compute nodes.
\end{tcolorbox}
\end{center}

\noindent To enable all 3rd party builds available in \OHPC{} that are compatible with
the \IntelR{}~oneAPI classic compiler suite, issue the following:


% ohpc_command if [[ ${enable_opa} -eq 1 ]];then
% ohpc_indent 10
\begin{lstlisting}[language=bash,keywords={},upquote=true,keepspaces]
# Optionally, choose the Omni-Path enabled build for MVAPICH2. Otherwise, skip to retain IB variant
[sms](*\#*) (*\install*) mvapich2-psm2-intel-ohpc
\end{lstlisting}
% ohpc_indent 5
% ohpc_command fi

%\iftoggle{isSLES_ww_slurm_x86}{\clearpage}

\begin{lstlisting}[language=bash,keywords={},upquote=true,keepspaces]
# Install 3rd party libraries/tools meta-packages built with Intel toolchain
[sms](*\#*) (*\install*) openmpi5-pmix-intel-ohpc
[sms](*\#*) (*\install*) ohpc-intel-serial-libs
[sms](*\#*) (*\install*) ohpc-intel-geopm
[sms](*\#*) (*\install*) ohpc-intel-io-libs
[sms](*\#*) (*\install*) ohpc-intel-perf-tools
[sms](*\#*) (*\install*) ohpc-intel-python3-libs
[sms](*\#*) (*\install*) ohpc-intel-mpich-parallel-libs
[sms](*\#*) (*\install*) ohpc-intel-mvapich2-parallel-libs
[sms](*\#*) (*\install*) ohpc-intel-openmpi5-parallel-libs
[sms](*\#*) (*\install*) ohpc-intel-impi-parallel-libs
\end{lstlisting}
% ohpc_indent 0
% ohpc_command fi
% end_ohpc_run



\clearpage
\section{Resource Manager Startup} \label{sec:rms_startup}
In section \S\ref{sec:basic_install}, the \SLURM{} resource manager was installed
and configured for use on both the {\em master} host and {\em compute} node
instances. With the cluster nodes up and functional, we can now startup the
resource manager services in preparation for running user jobs. Generally, this
is a two-step process that requires starting up the controller daemons on the {\em
 master} host and the client daemons on each of the {\em compute} hosts.
%Since the {\em compute} hosts were booted into an image that included the SLURM client
%components, the daemons should already be running on the {\em compute}
%hosts. 
Note that \SLURM{} leverages the use of the {\em munge} library to provide
authentication services and this daemon also needs to be running on all hosts
within the resource management pool. 
%The munge daemons should already
%be running on the {\em compute} subsystem at this point, 
The following commands can be used to startup the necessary services to support
resource management under \SLURM{}.

%\iftoggle{isCentOS}{\clearpage}

% Allow for optional sleep to wait for nodes to provision when using install
% script


% begin_ohpc_run
% ohpc_comment_header Allow for optional sleep to wait for provisioning to complete
% ohpc_command sleep ${provision_wait}
% end_ohpc_run

% begin_ohpc_run
% ohpc_comment_header Resource Manager Startup \ref{sec:rms_startup}
\begin{lstlisting}[language=bash,keywords={}]
# Start munge and slurm controller on master host
[sms](*\#*) systemctl enable munge
[sms](*\#*) systemctl enable slurmctld
[sms](*\#*) systemctl start munge
[sms](*\#*) systemctl start slurmctld

# Start slurm clients on compute hosts
[sms](*\#*) pdsh -w ${compute_prefix}[1-${num_computes}] systemctl start munge
[sms](*\#*) pdsh -w ${compute_prefix}[1-${num_computes}] systemctl start slurmd
\end{lstlisting}
% end_ohpc_run

%%% In the default configuration, the {\em compute} hosts will be initialized in an
%%% {\em unknown} state. To place the hosts into production such that they are
%%% eligible to schedule user jobs, issue the following:

%%% % begin_ohpc_run
%%% \begin{lstlisting}[language=bash]
%%% [sms](*\#*) scontrol update partition=normal state=idle
%%% \end{lstlisting}
%%% % end_ohpc_run



\section{Run a Test Job} \label{sec:test_job}
With the resource manager enabled for production usage, users should now be
able to run jobs. To demonstrate this, we will add a ``test'' user on the {\em master}
host that can be used to run an example job.

% begin_ohpc_run
\begin{lstlisting}[language=bash,keywords={}]
[sms](*\#*) useradd -m test
\end{lstlisting}
% end_ohpc_run

Next, the user's credentials need to be distributed across the cluster.
\xCAT{}'s \texttt{xdcp} has a merge functionality that adds new entries into
credential files on {\em compute} nodes:

% begin_ohpc_run
\begin{lstlisting}[language=bash,keywords={}]
# Create a sync file for pushing user credentials to the nodes
[sms](*\#*) echo "MERGE:" > syncusers
[sms](*\#*) echo "/etc/passwd -> /etc/passwd" >> syncusers
[sms](*\#*) echo "/etc/group -> /etc/group"       >> syncusers
[sms](*\#*) echo "/etc/shadow -> /etc/shadow" >> syncusers
# Use xCAT to distribute credentials to nodes
[sms](*\#*) xdcp compute -F syncusers
\end{lstlisting}
% end_ohpc_run

\nottoggle{isxCATstateful}{Alternatively, the \texttt{updatenode compute -f} command
can be used. This re-synchronizes (i.e. copies) all the files defined in the
\texttt{syncfile} setup in Section \ref{sec:file_import}. \\ }

~\\
\input{common/prun}

\iftoggle{isSLES_ww_slurm_x86}{\clearpage}
%\iftoggle{isxCAT}{\clearpage}

\subsection{Interactive execution}
To use the newly created ``test'' account to compile and execute the
application {\em interactively} through the resource manager, execute the
following (note the use of \texttt{prun} for parallel job launch which summarizes
the underlying native job launch mechanism being used):

\begin{lstlisting}[language=bash,keywords={}]
# Switch to "test" user
[sms](*\#*) su - test

# Compile MPI "hello world" example
[test@sms ~]$ mpicc -O3 /opt/ohpc/pub/examples/mpi/hello.c

# Submit interactive job request and use prun to launch executable
[test@sms ~]$ salloc -n 8 -N 2

[test@c1 ~]$ prun ./a.out

[prun] Master compute host = c1
[prun] Resource manager = slurm
[prun] Launch cmd = mpiexec.hydra -bootstrap slurm ./a.out

 Hello, world (8 procs total)
    --> Process #   0 of   8 is alive. -> c1
    --> Process #   4 of   8 is alive. -> c2
    --> Process #   1 of   8 is alive. -> c1
    --> Process #   5 of   8 is alive. -> c2
    --> Process #   2 of   8 is alive. -> c1
    --> Process #   6 of   8 is alive. -> c2
    --> Process #   3 of   8 is alive. -> c1
    --> Process #   7 of   8 is alive. -> c2
\end{lstlisting}

\begin{center}
\begin{tcolorbox}[]
The following table provides approximate command equivalences between SLURM and
OpenPBS:

\vspace*{0.15cm}
\input common/rms_equivalence_table
\end{tcolorbox}
\end{center}
\nottoggle{isCentOS}{\clearpage}

\iftoggle{isCentOS}{\clearpage}

\subsection{Batch execution}

For batch execution, \OHPC{} provides a simple job script for reference (also
housed in the \path{/opt/ohpc/pub/examples} directory. This example script can
be used as a starting point for submitting batch jobs to the resource manager
and the example below illustrates use of the script to submit a batch job for
execution using the same executable referenced in the previous interactive example.

\begin{lstlisting}[language=bash,keywords={}]
# Copy example job script
[test@sms ~]$ cp /opt/ohpc/pub/examples/slurm/job.mpi .

# Examine contents (and edit to set desired job sizing characteristics)
[test@sms ~]$ cat job.mpi
#!/bin/bash

#SBATCH -J test               # Job name
#SBATCH -o job.%j.out         # Name of stdout output file (%j expands to %jobId)
#SBATCH -N 2                  # Total number of nodes requested
#SBATCH -n 16                 # Total number of mpi tasks #requested
#SBATCH -t 01:30:00           # Run time (hh:mm:ss) - 1.5 hours

# Launch MPI-based executable

prun ./a.out

# Submit job for batch execution
[test@sms ~]$ sbatch job.mpi
Submitted batch job 339
\end{lstlisting}

\begin{center}
\begin{tcolorbox}[]
\small
The use of the \texttt{\%j} option in the example batch job script shown is a convenient
way to track application output on an individual job basis. The \texttt{\%j} token
is replaced with the \SLURM{} job allocation number once assigned (job~\#339 in
this example).
\end{tcolorbox}
\end{center}




\clearpage
\appendix
%\section*{Appendices}
{\bf \LARGE \centerline{Appendices}} \vspace*{0.2cm}

\addcontentsline{toc}{section}{Appendices}
\renewcommand{\thesubsection}{\Alph{subsection}}

\subsection{Installation Template}  \label{appendix:template_script}

This appendix highlights the availability of a companion installation script
that is included with \OHPC{} documentation. This script, when combined with
local site inputs, can be used to implement a starting recipe for
bare-metal system installation and configuration. This template script is used
during validation efforts to test cluster installations and is provided as a
convenience for administrators as a starting point for potential site
customization.

\vspace*{0.1cm}

\begin{center}
\begin{tcolorbox}[]
\small Note that the template script provided is intended for use during
initial installation and is not designed for repeated execution.  If
modifications are required after using the script initially, we recommend
running the relevant subset of commands interactively.
\end{tcolorbox}
\end{center}

The template script relies on the use of a simple text file to define local
site variables that were outlined in \S\ref{sec:inputs}. By default, the
template installation script attempts to use local variable settings sourced from
the \path{/opt/ohpc/pub/doc/recipes/vanilla/input.local} file, however, this
choice can be overridden by the use of the \texttt{\$\{OHPC\_INPUT\_LOCAL\}}
environment variable. The template install script is intended for execution on
the SMS {\em master} host and is installed as part of the \texttt{docs-ohpc}
package into \path{/opt/ohpc/pub/doc/recipes/vanilla/recipe.sh}. After
enabling the \OHPC{} repository and reviewing the guide for additional
information on the intent of the commands, the general starting approach for
using this template is as follows:

\begin{enumerate}
\item Install the \texttt{docs-ohpc} package

\begin{lstlisting}[language=bash,keywords={}]
[sms](*\#*) (*\install*) docs-ohpc
\end{lstlisting}

\item Copy the provided template input file to use as a starting point to
  define local site settings:
\begin{lstlisting}[language=bash,keywords={},literate={OSVER}{\baseosshort{}}1]
[sms](*\#*) cp /opt/ohpc/pub/doc/recipes/OSVER/input.local input.local
\end{lstlisting}

\item Update \path{input.local} with desired settings

\item Copy the template installation script which contains command-line
  instructions culled from this guide.

\begin{lstlisting}[language=bash,keywords={},basicstyle=\fontencoding{T1}\footnotesize\ttfamily,literate={OSVER}{\baseosshort{}}1
    {ARCH}{\arch{}}1 {PROV}{\MakeLowercase{\provisioner{}}}1
    {RMS}{\rmsshort{}}1 {-}{-}1]
[sms](*\#*) cp -p /opt/ohpc/pub/doc/recipes/OSVER/ARCH/PROV/RMS/recipe.sh .
\end{lstlisting}

\item Review and edit \path{recipe.sh} to suite.

\item Use environment variable to define local input file and execute
  \path{recipe.sh} to perform a local installation.

\begin{lstlisting}[language=bash,keywords={}]
[sms](*\#*) export OHPC_INPUT_LOCAL=./input.local
[sms](*\#*) ./recipe.sh
\end{lstlisting}
\end{enumerate}

\subsection{Upgrading OpenHPC Packages}  \label{appendix:upgrade}


As newer \OHPC{} releases are made available, users are encouraged to upgrade
their locally installed packages against the latest repository versions to
obtain access to bug fixes and newer component versions. This can be
accomplished with the underlying package manager as \OHPC{} packaging maintains
versioning state across releases. Also, package builds available from the
\OHPC{} repositories have ``\texttt{-ohpc}'' appended to their names so that
wild cards can be used as a simple way to obtain updates. The following general
procedure highlights a method for upgrading existing installations.
When upgrading from a minor release older than v\OHPCVerTree{}, you will first
need to update your local \OHPC{} repository configuration to point against the
v\OHPCVerTree{} release (or update your locally hosted mirror). Refer to
\S\ref{sec:enable_repo} for more details on enabling the latest
repository. In contrast, when upgrading between micro releases on the same
branch (e.g. from v\OHPCVerTree{} to \OHPCVerTree{}.2), there is no need to
adjust local package manager configurations when using the public repository as
rolling updates are pre-configured.

\begin{enumerate*}
\item (Optional) Ensure repo metadata is current (on head node and in chroot
  location(s)). Package managers will naturally do this on their own over time,
  but if you are wanting to access updates immediately after a new release,
  the following can be used to sync to the latest.

\begin{lstlisting}[language=bash,keywords={}]
[sms](*\#*)  (*\clean*)
[sms](*\#*)  (*\chrootclean*)
\end{lstlisting}

\item Upgrade master (SMS) node

\begin{lstlisting}[language=bash,keywords={}]
[sms](*\#*)  (*\upgrade*) "*-ohpc"

# Any new Base OS provided dependencies can be installed by
# updating the ohpc-base metapackage
[sms](*\#*)  (*\upgrade*) "ohpc-base"
\end{lstlisting}

\item Upgrade packages in compute image

\begin{lstlisting}[language=bash,keywords={}]
[sms](*\#*)  (*\chrootupgrade*) "*-ohpc"

# Any new compute-node Base OS provided dependencies can be installed by
# updating the ohpc-base-compute metapackage
[sms](*\#*)  (*\chrootupgrade*) "ohpc-base-compute"
\end{lstlisting}

\item Rebuild image(s)

\iftoggleverb{isWarewulf}
\begin{lstlisting}[language=bash,keywords={}]
[sms](*\#*) wwvnfs --chroot $CHROOT
\end{lstlisting}
\fi

\iftoggleverb{isxCAT}
\begin{lstlisting}[language=bash,keywords={},basicstyle=\fontencoding{T1}\fontsize{8.0}{10}\ttfamily,
    literate={-}{-}1 {BOSVER}{\baseos{}}1 {ARCH}{\arch{}}1]
[sms](*\#*) packimage BOSVER-x86_64-netboot-compute
\end{lstlisting}
\fi

\end{enumerate*}

\noindent In the case where packages were upgraded within the chroot compute image,
you will need to reboot the compute nodes when convenient to enable the
changes.

%%% \subsubsection{New component variants}
%%%
%%% As newer variants of key compiler/MPI stacks are released, \OHPC{} will
%%% periodically add toolchains enabling the latest variant. To stay consistent
%%% throughout the build hierarchy, minimize recompilation requirements for existing
%%% binaries, and allow for multiple variants to coexist, unique delimiters are
%%% used to distinguish RPM package names and module hierarchy.
%%%
%%% In the case of a fresh install, \OHPC{} recipes default to installation of the
%%% latest toolchains available in a given release branch. However, if upgrading a
%%% previously installed system, administrators can {\em opt-in} to enable new
%%% variants as they become available. To illustrate this point, consider the
%%% previous \OHPC{} 1.3.5 release as an example which contained GCC 7.3.0
%%% along with runtimes and libraries compiled with this toolchain.  In the case
%%% where an admin would like to enable the newer {``gnu8''} toolchain,
%%% installation of these additions is simplified with the use of \OHPC{}'s
%%% meta-packages (see Table~\ref{table:groups} in Appendix
%%% \ref{appendix:manifest}).  The following example illustrates adding the
%%% complete ``gnu8'' toolchain.  Note that we leverage the convenience
%%% meta-packages containing MPI-dependent builds, and we also update the
%%% modules environment to make it the default.
%%%
%%% \begin{lstlisting}[language=bash,keywords={}]
%%% # Install GCC 8.x-compiled meta-packages with dependencies
%%% [sms](*\#*)  (*\install*) ohpc-gnu8-perf-tools \
%%%                          ohpc-gnu8-io-libs \
%%%                          ohpc-gnu8-python-libs \
%%%                          ohpc-gnu8-runtimes \
%%%                          ohpc-gnu8-serial \
%%%                          ohpc-gnu8-parallel-libs
%%%
%%% # Update default environment
%%% [sms](*\#*) (*\remove*) lmod-defaults-gnu7-openmpi3-ohpc
%%% [sms](*\#*) (*\install*) lmod-defaults-gnu8-openmpi3-ohpc
%%%
%%% \end{lstlisting}
%%%
%%%

\clearpage
\subsection{Integration Test Suite}  \label{appendix:test_suite}

This appendix details the installation and basic use of the integration test
suite used to support \OHPC{} releases. This suite is not intended to replace
the validation performed by component development teams, but is instead,
devised to confirm component builds are functional and interoperable within
the modular \OHPC{} environment.
The test suite is generally organized by components and the \OHPC{} CI workflow
relies on running the full suite using \href{https://jenkins.io}{\color{blue}{Jenkins}} to test
multiple OS configurations and installation recipes.
%Each \OHPC{} component is equipped with a set of scripts and applications
%to test the integration of these components in a Jenkins CI
%environment.
To facilitate customization and running of the test suite locally, we
provide these tests in a standalone RPM.

\begin{lstlisting}
[sms](*\#*) (*\install*) test-suite-ohpc
\end{lstlisting}

The RPM installation creates a user named \texttt{ohpc-test} to house the test
suite and provide an isolated environment for execution.  Configuration of the
test suite is done using standard \GNU{} autotools semantics and the
\href{https://jenkins.io}{\color{blue}{BATS}} shell-testing framework is used
to execute and log a number of individual unit tests.  Some tests require
privileged execution, so a different combination of tests will be enabled
depending on which user executes the top-level \texttt{configure}
script. Non-privileged tests requiring execution on one or more compute nodes are
submitted as jobs through the \rms{} resource manager. The tests are further
divided into ``short'' and ``long'' run categories. The short run configuration
is a subset
of approximately 180 tests to demonstrate basic functionality of key components
(e.g. MPI stacks) and should complete in 10-20 minutes. The long run (around
1000 tests) is comprehensive and can take an hour or more to complete.

Most components can be tested individually, but a default configuration is
setup to enable collective testing. To test an isolated component, use the
\texttt{configure} option to disable all tests, then re-enable the desired test
to run. The \texttt{--help} option to \texttt{configure} will display all
possible tests. By default, the test suite will endeavor to run tests for
multiple MPI stacks where applicable. To restrict tests to only a subset of MPI
families, use the \texttt{--with-mpi-families} option
(e.g. \texttt{--with-mpi-families="openmpi4"}). Example output is shown below
(some output is omitted for the sake of brevity).

\begin{lstlisting}[literate={RMS}{\rms{}}1 {ARCH}{\arch{}}1]
[sms](*\#*) su - ohpc-test
[test@sms ~]$ cd tests
[test@sms ~]$ ./configure --disable-all --enable-fftw
checking for a BSD-compatible install... /bin/install -c
checking whether build environment is sane... yes
...
---------------------------------------------- SUMMARY ---------------------------------------------

Package version............... : test-suite-2.0.0

Build user.................... : ohpc-test
Build host.................... : sms001
Configure date................ : 2020-10-05 08:22
Build architecture............ : ARCH
Compiler Families............. : gnu9
MPI Families.................. : mpich mvapich2 openmpi4
Python Families............... : python3
Resource manager ............. : RMS
Test suite configuration...... : short
...
Libraries:
    Adios .................... : disabled
    Boost .................... : disabled
    Boost MPI................. : disabled
    FFTW...................... : enabled
    GSL....................... : disabled
    HDF5...................... : disabled
    HYPRE..................... : disabled
...
\end{lstlisting}

\iftoggle{isCentOS_ww_pbs_aarch}{\clearpage}

Many \OHPC{} components exist in multiple flavors to support multiple compiler
and MPI runtime permutations, and the test suite takes this in to account by
iterating through these combinations by default. If \texttt{make check} is
executed from the top-level test directory, all configured compiler and MPI
permutations of a library will be exercised. The following highlights the
execution of the FFTW related tests that were enabled in the previous step.

\begin{lstlisting}[literate={RMS}{\rms{}}1]
[test@sms ~]$ make check
make --no-print-directory check-TESTS
PASS: libs/fftw/ohpc-tests/test_mpi_families
============================================================================
Testsuite summary for test-suite 2.0.0
============================================================================
# TOTAL: 1
# PASS:  1
# SKIP:  0
# XFAIL: 0
# FAIL:  0
# XPASS: 0
# ERROR: 0
============================================================================
[test@sms ~]$ cat libs/fftw/tests/family-gnu*/rm_execution.log
1..3
ok 1 [libs/FFTW] Serial C binary runs under resource manager (RMS/gnu9/mpich)
ok 2 [libs/FFTW] MPI C binary runs under resource manager (RMS/gnu9/mpich)
ok 3 [libs/FFTW] Serial Fortran binary runs under resource manager (RMS/gnu9/mpich)
PASS rm_execution (exit status: 0)
1..3
ok 1 [libs/FFTW] Serial C binary runs under resource manager (RMS/gnu9/mvapich2)
ok 2 [libs/FFTW] MPI C binary runs under resource manager (RMS/gnu9/mvapich2)
ok 3 [libs/FFTW] Serial Fortran binary runs under resource manager (RMS/gnu9/mvapich2)
PASS rm_execution (exit status: 0)
1..3
ok 1 [libs/FFTW] Serial C binary runs under resource manager (RMS/gnu9/openmpi4)
ok 2 [libs/FFTW] MPI C binary runs under resource manager (RMS/gnu9/openmpi4)
ok 3 [libs/FFTW] Serial Fortran binary runs under resource manager (RMS/gnu9/openmpi4)
PASS rm_execution (exit status: 0)
\end{lstlisting}

\input{common/customization_appendix_centos}
%\newcommand{\firstColWidth}{2.3cm}
%\newcommand{\secondColWidth}{1.25cm}

\clearpage

\definecolor{Gray}{gray}{0.5}
\newcommand{\captionSpace}{-0.15cm}
\newcommand{\tabSpaceBot}{1.0cm}
\captionsetup{justification=raggedright,singlelinecheck=false}

\subsection{Package Manifest} \label {appendix:manifest}

\vspace*{0.25cm}
This appendix provides a summary of available meta-package groupings and all of
the individual RPM packages that are available as part of this \OHPC{}
release. The meta-packages provide a mechanism to group related collections of
RPMs by functionality and provide a convenience mechanism for installation.  A
list of the available meta-packages and a brief description is presented in
Table~\ref{table:groups}.

\vspace*{1.25cm}
\begin{table}[h!]
\caption{\bf Available \OHPC{} Meta-packages} \vspace*{\captionSpace{}}
\label{table:groups}
\input manifest/patterns
\end{table}

%% % meta-packages (2)
%% \begin{table}[h!]
%% \caption*{Table~\ref{table:groups} (cont): {\bf Available \OHPC{} Meta-packages} \vspace*{\captionSpace{}} }
%% \input manifest/patterns2
%% \end{table}

\iftoggleverb{isx86}
% meta-packages (3)
\begin{table}[h!]
\caption*{Table~\ref{table:groups} (cont): {\bf Available \OHPC{} Meta-packages} \vspace*{\captionSpace{}} }
\input manifest/patterns3
\end{table}

\fi

\clearpage
What follows next in this Appendix is a series of tables that summarize the
underlying RPM packages available in this \OHPC{} release. These packages are
organized by groupings based on their general functionality and each table
provides information for the specific RPM name, version, brief summary, and the
web URL where additional information can be obtained for the component. Note
that many of the 3rd party community libraries that are pre-packaged
with \OHPC{} are built using multiple compiler and MPI families. In these cases,
the RPM package name includes delimiters identifying the development
environment for which each package build is targeted.  Additional information
on the \OHPC{} package naming scheme is presented in \S\ref{sec:3rdparty}.
The relevant package groupings and associated Table references are as follows:

\vspace*{0.1cm}

\begin{itemize*}
\item Administrative tools (Table~\ref{table:admin})
\iftoggleverb{isWarewulf}
\item Provisioning (Table~\ref{table:provisioning})
\fi
\item Resource management (Table~\ref{table:rms})
\item Compiler families (Table~\ref{table:compiler-families})
\item MPI families (Table~\ref{table:mpi-families})
\item Development tools (Table~\ref{table:dev-tools})
\item Performance analysis tools (Table~\ref{table:perf-tools})
\iftoggleverb{isCentOS_x86}
\item Lustre (Table~\ref{table:lustre})
\fi

%\item Distro support packages and dependencies (Table~\ref{table:distro-packages})
\item IO Libraries (Table~\ref{table:io-libs})
\item Runtimes (Table~\ref{table:runtimes})
\item Serial/Threaded Libraries (Table~\ref{table:serial-libs})
\item Parallel Libraries (Table~\ref{table:parallel-libs})
\end{itemize*}


\newcommand{\firstColWidth}{5.3cm}
\newcommand{\secondColWidth}{1.25cm}

\vspace*{1.0cm}
\urlstyle{same}

% Administration Tools
\begin{table}[h]
\caption{\bf Administrative Tools} \vspace*{\captionSpace{}} \label{table:admin}
\input manifest/admin
\end{table}
\vspace*{0.5cm}

\renewcommand{\firstColWidth}{6.25cm}
\renewcommand{\secondColWidth}{1.25cm}

% Provisioning
\iftoggleverb{isWarewulf}
\begin{table}[h!]
\caption{\bf Provisioning} \vspace*{\captionSpace{}} \label{table:provisioning}
\input manifest/provisioning
\vspace*{\tabSpaceBot{}}
\end{table}
\fi

\renewcommand{\firstColWidth}{4.1cm}
\renewcommand{\secondColWidth}{1.8cm}

% Resource Management
\begin{table}[h!]
\caption{\bf Resource Management} \vspace*{\captionSpace{}} \label{table:rms}
\input manifest/rms
\vspace*{\tabSpaceBot{}}
\end{table}

\renewcommand{\firstColWidth}{4.7cm}

% Compiler Families
\begin{table}[h!]
\caption{\bf Compiler Families} \vspace*{\captionSpace{}} \label{table:compiler-families}
\input manifest/compiler-families
\vspace*{\tabSpaceBot{}}
\end{table}

\renewcommand{\firstColWidth}{4.7cm}

% MPI Families
\begin{table}[h!]
\caption{\bf MPI Families / Communication Libraries} \vspace*{\captionSpace{}} \label{table:mpi-families}
\input manifest/mpi-families
\vspace*{\tabSpaceBot{}}
\end{table}

\renewcommand{\firstColWidth}{5.6cm}
\renewcommand{\secondColWidth}{1.5cm}

% Development Tools
\begin{table}[h!]
\caption{\bf Development Tools} \vspace*{\captionSpace{}} \label{table:dev-tools}
\input manifest/dev-tools
\vspace*{\tabSpaceBot{}}
\end{table}

%%% % Development Tools (2)
%%% \begin{table}[h!]
%%% \caption*{Table~\ref{table:dev-tools} (cont): {\bf Development Tools} \vspace*{\captionSpace{}} }
%%% \input manifest/dev-tools2
%%% \vspace*{\tabSpaceBot{}}
%%% \end{table}

\renewcommand{\firstColWidth}{4.5cm}
\renewcommand{\secondColWidth}{1.7cm}

% Perf Tools (1)
\begin{table}[h!]
\caption{\bf Performance Analysis Tools} \vspace*{\captionSpace{}} \label{table:perf-tools}
\input manifest/perf-tools
\vspace*{\tabSpaceBot{}}
\end{table}

% Perf Tools (2)
\begin{table}[h!]
\caption*{Table~\ref{table:perf-tools} (cont): {\bf Performance Analysis Tools} \vspace*{\captionSpace{}} }
\input manifest/perf-tools2
\vspace*{\tabSpaceBot{}}
\end{table}

% Perf Tools (3)
\begin{table}[h!]
\caption*{Table~\ref{table:perf-tools} (cont): {\bf Performance Analysis Tools} \vspace*{\captionSpace{}} }
\input manifest/perf-tools3
\vspace*{\tabSpaceBot{}}
\end{table}

%%% % Perf Tools (4)
%%% \begin{table}[h!]
%%% \caption*{Table~\ref{table:perf-tools} (cont): {\bf Performance Analysis Tools} \vspace*{\captionSpace{}} }
%%% \input manifest/perf-tools4
%%% \vspace*{\tabSpaceBot{}}
%%% \end{table}
%%%
%%% % Distro Packages
%%% \begin{table}[h!]
%%% \caption{\bf Distro Support Packages/Dependencies} \vspace*{\captionSpace{}} \label{table:distro-packages}
%%% \input manifest/distro-packages
%%% \vspace*{\tabSpaceBot{}}
%%% \end{table}



% Lustre
\begin{table}[h!]
\caption{\bf Lustre} \vspace*{\captionSpace{}} \label{table:lustre}
\input manifest/lustre
\vspace*{\tabSpaceBot{}}
\end{table}

\renewcommand{\firstColWidth}{5.1cm}
\renewcommand{\secondColWidth}{1.4cm}

% IO Libs (1)
\begin{table}[h!]
\caption{\bf IO Libraries} \vspace*{\captionSpace{}} \label{table:io-libs}
\input manifest/io-libs
\vspace*{\tabSpaceBot{}}
\end{table}

% IO Libs (2)
\begin{table}[h!]
\caption*{Table~\ref{table:io-libs} (cont): {\bf IO Libraries} \vspace*{\captionSpace{}} }
\input manifest/io-libs2
\vspace*{\tabSpaceBot{}}
\end{table}

%%% % IO Libs (3)
%%% \begin{table}[h!]
%%% \caption*{Table~\ref{table:io-libs} (cont): {\bf IO Libraries} \vspace*{\captionSpace{}} }
%%% \input manifest/io-libs3
%%% \vspace*{\tabSpaceBot{}}
%%% \end{table}
%%%
%%% %\renewcommand{\firstColWidth}{4.5cm}
%%% %\renewcommand{\secondColWidth}{1.5cm}

% Runtimes
\clearpage
\begin{table}[h!]
\caption{\bf Runtimes} \vspace*{\captionSpace{}} \label{table:runtimes}
\input manifest/runtimes
\vspace*{\tabSpaceBot{}}
\end{table}

% Serial libs
\begin{table}[h!]
\caption{\bf Serial/Threaded Libraries} \vspace*{\captionSpace{}} \label{table:serial-libs}
\input manifest/serial-libs
\vspace*{\tabSpaceBot{}}
\end{table}

% Parallel libs (1)
\begin{table}[h!]
\caption{\bf Parallel Libraries} \vspace*{\captionSpace{}} \label{table:parallel-libs}
\input manifest/parallel-libs
\vspace*{\tabSpaceBot{}}
\end{table}

% Parallel libs (2)
\begin{table}[h!]
\caption*{Table~\ref{table:parallel-libs} (cont): {\bf Parallel Libraries} \vspace*{\captionSpace{}} }
\input manifest/parallel-libs2
\vspace*{\tabSpaceBot{}}
\end{table}

% Parallel libs (3)
\begin{table}[h!]
\caption*{Table~\ref{table:parallel-libs} (cont): {\bf Parallel Libraries} \vspace*{\captionSpace{}} }
\input manifest/parallel-libs3
\vspace*{\tabSpaceBot{}}
\end{table}
%%%
%%% % Parallel libs (4)
%%% \begin{table}[h!]
%%% \caption*{Table~\ref{table:parallel-libs} (cont): {\bf Parallel Libraries} \vspace*{\captionSpace{}} }
%%% \input manifest/parallel-libs4
%%% \vspace*{\tabSpaceBot{}}
%%% \end{table}
%%%
%%%




\input{common/signature}


\end{document}

